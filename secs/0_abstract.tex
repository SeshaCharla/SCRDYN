\section*{Abstract}
$\indent$ The existing approach to modelling the reaction flows of diesel engine SCR-ASC system uses a sequence of CSTRs
(n-cell models) to approximate the PDE's that arise from the plug-flow reactions by discretizing in space. This model
introduces a causality reversal in computation of reaction rates in each cell as they become function of outlet
concentrations, i.e, the concentrations of the products after the reaction. This approximation breaks down when the
n-cell model is reduced down to a single-cell in order to have minimal realization of the dynamics. Thus, the CSTR
modelling approach requires multiple cells to appropriately capture the reaction dynamics increasing the number of
states and parameters that need estimation. Further, it is observed that the sampling time for the measurement signals
is larger than the actual residence time of the reactants in the chamber. Thus, the measurement signals only capture the
averaged effects of the reactions happening in multiple residence times.

Considering these insights, the present work aims to develop a reduced order nonlinear discrete model from first
principles that can be parametrized into an identifiable model. Instead of discretizing the plug-flow reaction in space,
we first discretize the system in time considering the interplay between the residence time and sampling time. Then the
reaction process dynamics in the SCR-ASC chambers is lumped and reduced to a few significant reactions whose effect on
the time evolution of the measurement signals at the inlet and outlet of the chamber is modelled as a set of difference
equations based on the molar conservation of species. These equations are parametrized as the function of inputs and
states using approximate models for physical properties such as rate constant and density based on the operating
conditions of the system. Finally, an identifiable model for the $NO_x$ process dynamics is developed and validate with
the test-cell data.
