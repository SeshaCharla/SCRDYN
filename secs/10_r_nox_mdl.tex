\newpage
\section{Nonlinear recursive model with linear parameters for tailpipe $NO_x$ $\lr{\con{NO_x}^{out}}$ dynamics}

Let us define the amount of $NO_x$ reduced at the end of time-step $k$ as $\eta(k)$, i.e.,
\begin{align}
        \eta (k) &= u_1(k-1) - x_1(k)
        \label{eqn::eta_proc}
\end{align}

Rewriting equation \ref{eqn::nox_regression} to get the adsorbed ammonia:
\begin{align}
        x_3(k) \lrf{\pmb \phi_1^T(k) \pmb \theta_1} &= u_1(k)-x_1(k+1)  = \eta\lr{k+1}
        \label{eqn::eta_x3_1}\\
        %===
        \implies x_3(k+1) \lrf{\pmb \phi_1^T(k+1) \pmb \theta_1} &= u_1(k+1)-x_1(k+2)  = \eta\lr{k+2}
        \label{eqn::eta_x3_2}
\end{align}
Consider \ref{eqn::nh3_ads_regression}:
\begin{align*}
     x_3(k+1) &= x_3 \pmb \phi_3^T \pmb \theta_3 + \pmb \phi_{ur} \pmb \theta_\Gamma\\
     %===
     \text{Multiplying both sides by } & \lrf{\pmb \phi_1^T(k) \pmb \theta_1} \\
     %===
     \implies \lrf{\pmb \phi_1^T(k) \pmb \theta_1} x_3(k+1) &= \lrf{\pmb \phi_1^T(k) \pmb \theta_1} \lrf{x_3 \pmb \phi_3^T \pmb \theta_3 + \pmb \phi_{ur} \pmb \theta_\Gamma}\\
     %===
     &= \underbrace{\lrf{\lr{\pmb \phi_1^T(k) \pmb \theta_1}x_3(k)}}_{\eta(k+1)}
      \lrf{\pmb \phi_3^T \pmb \theta_3}
     + \lrf{\pmb \phi_1^T(k) \pmb \theta_1} \lrf{\pmb \phi_{ur} \pmb \theta_\Gamma}
     %===
     \qquad \lrb{\because \ref{eqn::eta_x3_1}}
\end{align*}
\begin{align}
        \therefore
        \lrf{\pmb \phi_1^T(k) \pmb \theta_1} x_3(k+1)
        &= \eta(k+1) \lrf{\pmb \phi_3^T \pmb \theta_3} + \lrf{\pmb \phi_1^T(k) \pmb \theta_1} \lrf{\pmb \phi_{ur} \pmb \theta_\Gamma}
        \label{eqn::r_nox_1}
\end{align}
Let,
\begin{align}
        f_{\phi_1}(k+1) &= \pmb \phi_1(k+1)^T \pmb \phi_1^{T+}(k)
        \label{eqn::f_phi_1}
\end{align}
Where, $\pmb \phi_1^{T+}$ be the Moore-Penrose pseudo inverse of $\pmb \phi_1^T(k)$. Thus,
\begin{align}
        f_{\phi_1}(k+1) \lrf{\pmb \phi_1^T(k) \pmb \theta_1} &= \pmb \phi_1(k+1)^T  \underbrace{\pmb \phi_1^{T+}(k) \pmb \phi_1^T(k)}_{\approx I_2} \pmb \theta_1
        = \pmb \phi_1^T(k+1) \pmb \theta_1
        \label{eqn::f_phi_prop}
\end{align}
Note that $f_{\phi_1}$ is a scalar. Multiplying $f_{\phi_1}(k+1)$ on both sides of equation (\ref{eqn::r_nox_1}):
\begin{align*}
      \underbrace{f_{\phi_1}(k+1) \lrf{\pmb \phi_1^T(k) \pmb \theta_1} x_3(k+1)}_{= \lrf{\pmb \phi_1^T(k+1) \pmb \theta_1} x_3(k+1)  = \eta (k+2)}
      &= f_{\phi_1}(k+1) \lrf{\eta(k+1) \lrf{\pmb \phi_3^T \pmb \theta_3} + \lrf{\pmb \phi_1^T(k) \pmb \theta_1} \lrf{\pmb \phi_{ur} \pmb \theta_\Gamma}} \\
      %===
      \implies \eta(k+2) &= \eta (k+1) f_{\phi_1}(k+1)\lrf{\pmb \phi_3^T (k) \pmb \theta_3}
                + \lrf{ \lr{\pmb \phi_1^T(k+1) \pmb \theta_1} \kron \lr{\pmb \phi^T_{ur}(k) \pmb \theta_\Gamma}}
        \quad \lrb{\because \ref{eqn::f_phi_prop}}
\end{align*}
Writing in one time step ahead form:
\begin{align}
        \eta (k+1) &=  \eta (k) f_{\phi_1}(k)\lrf{\pmb \phi_3^T (k-1) \pmb \theta_3}
                + \lrf{ \lr{\pmb \phi_1^T(k) \pmb \theta_1} \kron \lr{\pmb \phi^T_{ur}(k-1) \pmb \theta_\Gamma}}
        \label{eqn::r_nox_2}
\end{align}
%===
Expanding the terms and defining the following new regressor and parameter vectors,
\begin{align*}
        \text{Let, } \qquad &\\
        \pmb \theta_3 &= \bm{ 1 &
                                t_s \nu_u \pmb \theta_{ads}^T &
                                t_s \pmb \theta_{od}^T        &
                                t_s \pmb \theta_{scr}^T} ^T
                        = \bm{1 & \pmb \theta_{f1}}^T\\
        %==
        \text{and, } \qquad &\\
        \pmb \phi_3^T &= \bm{1 &
                   - \pmb \phi_{ur}^T &
                   - \pmb \phi^T  &
                   - u_1 \pmb \phi^T}   \\
        %===
        \implies \eta (k) f_{\phi_1}(k)\lrf{\pmb \phi_3^T (k-1)} &= \eta(k) f_{\phi_1}(k) - \pmb \phi_{f1}(k)\\
        %===
        \text{where, } \qquad &\\
        \pmb \phi_{f1}(k)
                &= \eta(k) f_{\phi_1}(k)
                \bm{ \pmb \phi_{ur}^T(k-1) &
                    \pmb \phi^T(k-1)  &
                    u_1 (k-1) \pmb \phi^T(k-1)}   \\
        \text{and, } \qquad &\\
        \pmb \theta_{f1}^T &= \bm{ t_s \nu_u \pmb \theta_{ads}^T &
                                t_s \pmb \theta_{od}^T        &
                                t_s \pmb \theta_{scr}^T} ^T
\end{align*}
We have the following properties of Kronecker product:
        \begin{align*}
                \lr{A \kron B}^T &= A^T \kron B^T \\
                \lr{A \kron B} \lr{C \kron D} &= \lr{AC} \kron \lr{BD}
        \end{align*}
Thus, Kronecker product can be rewritten into product of regressor and parameter vectors as:
\begin{align*}
        \lrf{\pmb \phi_1^T(k) \pmb \theta_1} \kron \lrf{\pmb \phi^T_{ur}(k-1) \pmb \theta_\Gamma} &=
             \lrf{ \pmb \phi_1^T(k) \kron \pmb \phi^T_{ur}(k-1) }
             \lrf{\pmb \theta_1 \kron \pmb \theta_\Gamma}\\
        \pmb \phi_{\Gamma 1} (k) &= \lrf{ \pmb \phi_1^T(k) \kron \pmb \phi^T_{ur}(k-1) } \\
        \pmb \theta_{\Gamma 1} &= \lrf{\pmb \theta_1 \kron \pmb \theta_\Gamma}
\end{align*}
The regressors coming from  of Kronecker products will have full rank as they involve the product of inputs at different time steps. Thus, they can be algorithmically computed and need not be simplified to a vector expression.

Substituting the above definitions into equation (\ref{eqn::r_nox_2}):
\begin{align}
         \eta(k+1) &=  \eta(k) f_{\phi_1}(k) -
                     \pmb \phi_{f1}^T(k) \pmb \theta_{f1} +
                     \pmb \phi_{\Gamma 1}^T(k) \pmb \theta_{\Gamma 1}
        \label{eqn::r_nox_3}
\end{align}
Let,
\begin{align}
        \pmb \phi_{NO_x} &= \bm{-\pmb \phi_{f1} & \pmb \phi_{\Gamma 1}}\\
        \pmb \theta_{NO_x} &= \bm{\pmb \theta_{f1} & \pmb \theta_{\Gamma 1}}
\end{align}
\begin{align}
          \eta (k+1) &=  \eta(k) f_{\phi_1}(k) +
                     \pmb \phi_{NO_x}^T (k) \pmb \theta_{NO_x}
        \label{eqn::r_nox_regression}
\end{align}

%%===
\subsection{Calculating $f_{\phi_1}$}

We have,
\begin{align*}
        f_{\phi_1}(k) &= \pmb \phi_1^T(k) \pmb \phi_1^{T+}(k-1)
\end{align*}
Calculating the Moore-Penrose pseudo inverse of $\pmb \phi_1^T$ using Tikhonov's regularization \cite{barata2012moore}.
\begin{align*}
        \pmb \phi_1^{T+} &= \lim_{\varepsilon \rightarrow 0}
                                 \lrf{\pmb \phi_1  \pmb \phi_1^T + \varepsilon I_2}^{-1} \pmb \phi_1 \\
        %===
        \pmb \phi_1^T &= \frac{u_1}{F} \bm{T & 1}\\
        %===
        \implies \lrf{\pmb \phi_1  \pmb \phi_1^T + \varepsilon I_2} &=  \frac{u_1^2}{F^2} \bm{T^2 + \varepsilon & T \\
                                                                                   T                 & 1 + \varepsilon}\\
        %===
        \implies \lrf{\pmb \phi_1  \pmb \phi_1^T + \varepsilon I_2}^{-1} &= \frac{F^2}{u_1^2}
                                                        \lr{\frac{1}{\varepsilon^2 + \varepsilon T^2 + \varepsilon}}
                                                                \bm{1+\varepsilon & -T \\
                                                                        -T      & T^2 + \varepsilon}\\
        %===
        \implies \lrf{\pmb \phi_1  \pmb \phi_1^T + \varepsilon I_2}^{-1} \pmb \phi_1 &=
                                \frac{F}{u_1} \lr{\frac{1}{\varepsilon^2 + \varepsilon T^2 + \varepsilon}}
                                \bm{1+\varepsilon & -T \\
                                     -T      & T^2 + \varepsilon}
                                \bm{T \\ 1}
        %===
        &= \frac{F}{u_1} \lr{\frac{1}{\varepsilon^2 + \varepsilon T^2 + \varepsilon}}
        \bm{T \varepsilon \\ \varepsilon}\\
        %===
        &= \frac{F}{u_1} \lr{\frac{1}{\varepsilon + T^2 + 1}}
        \bm{T \\ 1}\\
        %===
        \implies  \pmb \phi_1^{T+} &= \lim_{\varepsilon \rightarrow 0}
                                 \lrf{\pmb \phi_1  \pmb \phi_1^T + \varepsilon I_2}^{-1} \pmb \phi_1
        = \frac{F}{u_1} \lr{\frac{1}{T^2 + 1}}
        \bm{T \\ 1}
\end{align*}
Thus,
\begin{align*}
        f_{\phi_1}(k) &= \pmb\phi_1^T(k) \pmb \phi_1^{T+}(k-1)
                      = \frac{u_1(k)}{F(k)} \bm{T(k) & 1} \frac{F(k-1)}{u_1(k-1)}
                      \lr{\frac{1}{T(k-1)^2 + 1}} \bm{T(k-1) \\ 1}\\
\end{align*}
Hence,
\begin{align}
        f_{\phi_1}(k) &= \lr{\frac{u_1(k)}{u_1(k-1)}} \lr{\frac{F(k-1)}{F(k)}} \lr{\frac{T(k)T(k-1) + 1}{T(k-1)^2 + 1}}
\end{align}

\subsection{Approximating $\pmb \phi_{\Gamma 1}$}

We have,
\begin{align}
        \pmb \phi_{\Gamma 1} \pmb \theta_{\Gamma 1} &= \lrf{\pmb \phi_1 (k) \pmb \theta_1} \times \lrf{\pmb \phi_{ur}(k-1) \theta_{\Gamma}}\\
        %===
        \pmb \phi_1 (k)  \pmb \theta_{1}&= u_1(k) k_{s2v} k_{scr}(k) \tau (k) \qquad \lrb{\because \ref{eqn::nox_regression}}\\
        %===
        \pmb \phi_{ur}(k-1) \pmb \theta_{\Gamma} &= \Gamma \times t_s k_{ads}(k-1) \times \nu_0 \lr{\frac{u_{inj}(k-1)}{F(k-1)}}
        \qquad \lrb{\because \ref{eqn::nh3_ads_regression}}\\
        %===
\end{align}
\begin{multline}
        \implies  \pmb \phi_{\Gamma 1} \pmb \theta_{\Gamma 1} =
                \lrf{k_{s2v} \tau_0 \times \lr{\frac{u_1(k)}{F(k)}} \times k_{scr}(k)} \times
                \lrf{\Gamma t_s \nu_0 \times \lr{\frac{u_{inj}(k-1)}{F(k-1)}} \times k_{ads}(k-1)} \\
        = \Gamma t_s \nu_0 k_{s2v} \tau_0 \times \lr{\frac{u_1(k)}{F(k)}} \lr{\frac{u_2(k-1)}{F(k-1)}} \times  k_{scr}(k) k_{ads}(k-1)
\end{multline}

Consider the product of rate constants:
\begin{align*}
        k_{scr}(k) k_{ads}(k-1) &= A_{scr} e^{\frac{-E_{scr}}{R T(k)}} \times A_{ads} e^{\frac{-E_{ads}}{R T(k-1)}} \qquad \lrb{\because \text{Arrhenius Equation}}\\
        %===
        &= A_{scr}A_{ads} \times \exp \lrf{\frac{-E_{scr}}{R T(k)} - \frac{E_{ads}}{R T(k-1)}}
\end{align*}
Assuming the temperature doesn't change significantly within one time step, i.e., $T(k) \approx T(k-1)$:
\begin{align}
        k_{scr}(k) k_{ads}(k-1) &= A_{scr}A_{ads} \exp{\frac{-\lr{E_{scr}+E_{ads}}}{R T(k)}}\\
                                &= \pmb \phi^T(k) \pmb \theta_{scr/ads} \qquad \lrb{\because \ref{eqn::temp_k}}
\end{align}
%===
Thus,
\begin{align}
        \pmb \phi_{\Gamma 1}^T \pmb \theta_{\Gamma 1} &=
        \underbrace{ \lr{\frac{u_1(k)}{F(k)}} \lr{\frac{u_2(k-1)}{F(k-1)}} \times \pmb \phi^T(k)}
        _{\pmb \phi_{\Gamma 1}^T(k)}  \times
        \underbrace{ \Gamma t_s \nu_0 k_{s2v} \tau_0 \times \pmb \theta_{scr/ads}}
        _{\pmb \theta_{\Gamma 1}}
\end{align}

This approximation not only reduces  the total number of paremeters but also keeps $\pmb \phi_{\Gamma 1}$'s elements independent when the temperature is slowly varying or a constant.

\subsection{Algorithmically calculating the regressor $\pmb \phi_{NO_x}$}

Following equations need to be computed in sequence for getting $\pmb \phi_{NO_x}$:


\begin{align*}
        1)&& \pmb \phi(k) &= \bm{T(k) & 1}\\
        %===
        2)&& \pmb \phi_{f1}(k) &= \eta(k) \lr{\frac{F(k)}{u_1(k)}}
                        \bm{ \lr{\frac{u_{inj}(k-1)}{u_1(k-1)}} \pmb \phi^T(k)
                              & \lr{\frac{F (k-1)}{u_1(k-1)}} \pmb \phi^T(k)
                              & F (k-1) \pmb \phi^T(k)} \\
        %===
        3)&& \pmb \phi_{\Gamma 1}(k) &= \lr{\frac{u_1(k)}{F(k)}} \lr{\frac{u_2(k-1)}{F(k-1)}} \times \pmb \phi^T(k) \\
        %===
        4)&& \pmb \phi_{NO_x}(k) &= \bm{-\pmb \phi_{f1}(k) \\ \pmb \phi_{\Gamma 1} (k)}
\end{align*}

Keen observation of the terms of $\pmb \phi_{NO_x}$ reveals that there is a common factor $\lr{\frac{u_1(k)}{F(k)}}$ in all the elements (steps, 2 and 3). This adds up the error due to adding the same set of signals resulting in higher errors in the parameter estimates. This can be avoided by getting the common factor to the RHS of the regression equation. Let,
\begin{align}
        F_{u_1}(k) &= \lr{\frac{F(k)}{u_1(k)}}\\
        %===
        f_r(k) &= F_{u_1}(k) f_{\phi_1}(k)
                = \lr{\frac{F(k-1)}{u_1(k-1)}} \lr{\frac{T(k)T(k-1) + 1}{T(k-1)^2 + 1}}
                = F_{u_1}(k-1) \times \lr{\frac{T(k)T(k-1) + 1}{T(k-1)^2 + 1}}  \\
        %===
        \pmb \phi_{f_r} &= F_{u_1}(k) \pmb \phi_{f1} (k)
                         = \eta(k)
                        \bm{ \lr{\frac{u_{inj}(k-1)}{u_1(k-1)}} \pmb \phi^T(k)
                              & \lr{\frac{F (k-1)}{u_1(k-1)}} \pmb \phi^T(k)
                              & F (k-1) \pmb \phi^T(k)} \\
        %===
        \pmb \phi_{\Gamma r}(k) &= F_{u_1} (k) \pmb \phi_{\Gamma 1} (k)
                                 = \lr{\frac{u_2(k-1)}{F(k-1)}} \times \pmb \phi^T(k) \\
        %===
        \text{Thus, } \qquad &
        %===
        \pmb \phi_r (k)  = F_{u_1}(k) \pmb \phi_{NOx}(k)
                         = \bm{-\pmb \phi_{f_r}(k) \\ \pmb \phi_{\Gamma_r} (k)}
\end{align}

Thus, we get the new regression form of the process dynamics from (\ref{eqn::r_nox_regression}):
\begin{align*}
        \eta (k+1) &=  \eta(k) f_{\phi_1}(k) +
                     \pmb \phi_{NO_x}^T (k) \pmb \theta_{NO_x}\\
        % ===
        \text{Multiplying } F_{u_1}(k) & \text{ on both sides:}\\
        % ===
        F_{u_1}(k) \eta (k+1) &=  F_{u_1}(k) \eta(k) f_{\phi_1}(k) +
                                        F_{u_1}(k) \pmb \phi_{NO_x}^T (k) \pmb \theta_{NO_x}\\
        % ===
        \implies  F_{u_1}(k) \eta (k+1) &= \eta(k) f_r (k) + \pmb \phi_{NO_x}^T (k) \pmb \theta_{NO_x}
\end{align*}
Let,
\begin{align}
       y_r(k) &= F_{u1}(k) \eta (k+1) - \eta(k) f_r (k)
\end{align}
Thus, the equation (\ref{eqn::r_nox_regression}) can be re-written as:
\begin{align}
       y_r(k) &= \pmb \phi_r(k) \pmb \theta_{NO_x}
\end{align}

The algorithm for computing $\pmb \phi_r$ and $y_r$ would be:

\begin{align*}
        1)&& \pmb \phi(k) &= \bm{T(k) & 1} ^T\\
        %===
        2)&& f_{r}(k) &= \lr{\frac{F(k-1)}{u_1(k-1)}} \lr{\frac{T(k)T(k-1) + 1}{T(k-1)^2 + 1}}\\
        %===
        3)&& \pmb \phi_{fr}(k) &= \eta(k)
                                        \bm{ \lr{\frac{u_{inj}(k-1)}{u_1(k-1)}} \pmb \phi^T(k)
                                              & \lr{\frac{F (k-1)}{u_1(k-1)}} \pmb \phi^T(k)
                                              & F (k-1) \pmb \phi^T(k)} \\
        %===
        4)&& \pmb \phi_{\Gamma r}(k) &= \lr{\frac{u_2(k-1)}{F(k-1)}} \times \pmb \phi(k) \\
        %===
        5)&& \pmb \phi_{r}(k) &= \bm{-\pmb \phi_{fr}(k) \\ \pmb \phi_{\Gamma r} (k)} \\
        %===
        6)&& y_{r}(k) &= \lr{\frac{F(k)}{u_1(k)}} \eta (k+1) - \eta(k) f_r (k)
\end{align*}

\subsection{Elements of $\pmb \theta_{NO_x}$}

We have,
\begin{align*}
        \pmb \theta_{NO_x} &= \bm{\pmb \theta_{f1}^T, \pmb \theta_{\Gamma 1}^T}^T
\end{align*}

We have the component parameter vectors that make up the parameter vectors $\pmb \theta_{f1}, \pmb \theta_{\Gamma 1}$:

\begin{align*}
        \pmb \theta_{scr} &= \bm{\tau_0 m_{scr} &
                                \tau_0 c_{scr}}^T
        \\
        \pmb \theta_{od}  &= \bm{\tau_0 m_{od} &
                                \tau_0 c_{od}}^T
        \\
        \pmb \theta_{ads} &= \bm{\tau_0 m_{ads} &
                                \tau_0 c_{ads} }^T
        \\
        \pmb \theta_{ads, ur} &= \nu_u \pmb \theta_{ads}^T
        \\
        \pmb \theta_{\Gamma } &= n \Gamma \pmb \theta_{ads, ur}
        \\
        \pmb \theta_1 &= k_{s2v} \pmb \theta_{scr}
\end{align*}

Thus,

\begin{align*}
        \pmb \theta_{f1} &= n \bm{\pmb \theta_{ads, ur}^T & \pmb \theta_{od}^T & \pmb \theta_{scr}^T}^T_{(6 \times 1)}\\
        %===
        \pmb \theta_{\Gamma 1} &= \pmb \theta_1 \kron \pmb \theta_\Gamma = n \Gamma k_{s2v} \lrf{ \pmb \theta_{scr} \kron \pmb \theta{ads, ur}}\\
                        &= n \Gamma k_{s2v} \nu_u \tau_0^2
                                \bm{m_{scr} m_{ads} \\
                                    m_{scr} c_{ads} \\
                                    c_{scr} m_{ads} \\
                                    c_{scr} c_{ads}}_{(4 \times 1)}
\end{align*}
\begin{align}
        \pmb \theta_{NO_x} &= n \bm{\pmb \theta_{ads, ur}^T & \pmb \theta_{od}^T & \pmb \theta_{scr}^T
                                    &|& \Gamma k_{s2v} \nu_u \lr{\pmb \theta_{scr}^T \kron \pmb \theta_{ads}^T}}^T_{(10 \times 1)}
\end{align}

Thus, all the parameters are sensitive to changes in sampling rate. The last 4 parameters explicitly depend on the surface void concentration of the catalyst.

