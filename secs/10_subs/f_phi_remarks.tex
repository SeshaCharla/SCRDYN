\subsection{Calculating $f_{\phi_1}$}

We have,
\begin{align*}
        f_{\phi_1}(k) &= \pmb \phi_1^T(k) \pmb \phi_1^{T+}(k-1)
\end{align*}
Calculating the Moore-Penrose pseudo inverse of $\pmb \phi_1^T$ using Tikhonov's regularization \cite{barata2012moore}.
\begin{align*}
        \pmb \phi_1^{T+} &= \lim_{\varepsilon \rightarrow 0}
                                 \lrf{\pmb \phi_1  \pmb \phi_1^T + \varepsilon I_2}^{-1} \pmb \phi_1 \\
        %===
        \pmb \phi_1^T &= \frac{u_1}{F} \bm{T & 1}\\
        %===
        \implies \lrf{\pmb \phi_1  \pmb \phi_1^T + \varepsilon I_2} &=  \frac{u_1^2}{F^2} \bm{T^2 + \varepsilon & T \\
                                                                                   T                 & 1 + \varepsilon}\\
        %===
        \implies \lrf{\pmb \phi_1  \pmb \phi_1^T + \varepsilon I_2}^{-1} &= \frac{F^2}{u_1^2}
                                                        \lr{\frac{1}{\varepsilon^2 + \varepsilon T^2 + \varepsilon}}
                                                                \bm{1+\varepsilon & -T \\
                                                                        -T      & T^2 + \varepsilon}\\
        %===
        \implies \lrf{\pmb \phi_1  \pmb \phi_1^T + \varepsilon I_2}^{-1} \pmb \phi_1 &=
                                \frac{F}{u_1} \lr{\frac{1}{\varepsilon^2 + \varepsilon T^2 + \varepsilon}}
                                \bm{1+\varepsilon & -T \\
                                     -T      & T^2 + \varepsilon}
                                \bm{T \\ 1}
        %===
        &= \frac{F}{u_1} \lr{\frac{1}{\varepsilon^2 + \varepsilon T^2 + \varepsilon}}
        \bm{T \varepsilon \\ \varepsilon}\\
        %===
        &= \frac{F}{u_1} \lr{\frac{1}{\varepsilon + T^2 + 1}}
        \bm{T \\ 1}\\
        %===
        \implies  \pmb \phi_1^{T+} &= \lim_{\varepsilon \rightarrow 0}
                                 \lrf{\pmb \phi_1  \pmb \phi_1^T + \varepsilon I_2}^{-1} \pmb \phi_1
        = \frac{F}{u_1} \lr{\frac{1}{T^2 + 1}}
        \bm{T \\ 1}
\end{align*}
Thus,
\begin{align*}
        f_{\phi_1}(k) &= \pmb\phi_1^T(k) \pmb \phi_1^{T+}(k-1)
                      = \frac{u_1(k)}{F(k)} \bm{T(k) & 1} \frac{F(k-1)}{u_1(k-1)}
                      \lr{\frac{1}{T(k-1)^2 + 1}} \bm{T(k-1) \\ 1}\\
\end{align*}
Hence,
\begin{align}
        f_{\phi_1}(k) &= \lr{\frac{u_1(k)}{u_1(k-1)}} \lr{\frac{F(k-1)}{F(k)}} \lr{\frac{T(k)T(k-1) + 1}{T(k-1)^2 + 1}}
\end{align}


\itbf{Note:} This approximate inversion of the $\phi_1$ inherently assumes that the temperature changes slowly as compared to other intput and states which can be verified using the test-cell and the truck data. Further, this approximation can be used to reduce the size of the $\pmb \phi_{\Gamma1}$ ensuring that it is a full rank matrix.
