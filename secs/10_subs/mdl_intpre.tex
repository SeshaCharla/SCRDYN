\subsubsection{Physical Interpretation of the Model Parameters}
We have the parameter vector can be written in the portioned form as follows:
\begin{align*}
\bm{ t_s \nu_u \pmb \theta_{ads} \\
     t_s \pmb \theta_{od}        \\
     t_s \pmb \theta_{scr}      \\
     \Gamma t_s \nu_u k_{s2v} \tau_0 \pmb \theta_{scr/ads}
} \qquad \text{Where, } \qquad
\mat{
        ads & - & \text{Ammonia Adsorption} \\
        od  & - & \text{Ammonia oxidation and desorption} \\
        scr & - & \text{SCR reaction} \\
        scr/ads & - & \text{Ammonia storage change} \\
}
\end{align*}
Each of the partitions correspond to the rate constant of a particular reaction $i = ads, od, scr$ and $scr/ads$. The two components of the vector $\pmb \theta_i$ are the temperature dependent ($m$) and the temperature independent ($c$).

Intuition dictates that these vector should 'cluster' based on the aging level of the catalyst especially the ammonia adsorption and the ammonia storage change. Degreened catalyst will have a larger 'capacity' to adsorb gaseous ammonia thus the magnitude of the $t_s \nu_u \pmb \theta_{ads}$ vector should be higher than the aged catalyst. Similarly, the aged catalyst has a lower magnitude of $\Gamma$, the surface concentration of available voids, thus have a lower magnitude of $\Gamma t_s \nu_u k_{s2v} \tau_0 \pmb \theta_{scr/ads}$ than a degreened catalyst.
