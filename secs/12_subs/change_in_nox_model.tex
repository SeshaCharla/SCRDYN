\subsection{Modelling using average concentration change of $NO_x$ in a sample}
$\eta$ denote the change in concentration from the inlet to outlet in the $NO_x$ in one sample. This is the change in concentration in the exhaust due to the $NO_x$ reduction.
%
\begin{align}
        \eta(k) &= \con{NO_x}^{in}(k-1) - \con{NO_x}^{out}(k) = u_1(k-1) - x_1(k)
\end{align}
%
Thus, rewriting the $NO_x$ process dynamics (\ref{eqn::nox_avg}) interms of $\eta$, we have,
\begin{align}
        \eta(k+1) &= \tau(k) k_{s2v} k_{scr}(k) u_1(k) \sigma(k)\\
        %===
        \implies \sigma(k) &= \frac{\eta(k+1)}{\tau(k) k_{s2v} k_{scr}(k) u_1(k)}
        \label{eqn::sigma_elim}
\end{align}
%
The above equation (\ref{eqn::sigma_elim}) can be used to eliminate the unknown quantity $(\sigma)$ from equation (\ref{eqn::ads_avg}). We have,
\begin{multline}
        \sigma(k+1) = \sigma(k) - \sigma(k) t_s k_{ads}(k) \con{NH_3}^{in}(t)
                        - \sigma(k) t_s k_{scr}(k) u_1(k)
                        - \sigma(k) t_s k_{od}(k)
                        \\ + \Gamma(k) t_s k_ads(k) \con{NH_3}^{in}(k)
\end{multline}
writing the above equation for $\sigma(k)$:
\begin{multline*}
         \sigma(k) = \sigma(k-1)\lrf{1 -  t_s k_{ads}(k-1) \con{NH_3}^{in}(t-1)
                        -  t_s k_{scr}(k-1) u_1(k-1)
                        -  t_s k_{od}(k-1)}
                        \\ + \Gamma(k-1) t_s k_{ads}(k-1) \con{NH_3}^{in}(k-1)
\end{multline*}
Let,
\begin{align}
        \gamma_{proc}(k-1) &= \lrf{1 -  t_s k_{ads}(k-1) \con{NH_3}^{in}(t-1)
                        -  t_s k_{scr}(k-1) u_1(k-1)
                        -  t_s k_{od}(k-1)}
\end{align}
Using equation (\ref{eqn::sigma_elim}),
\begin{multline*}
        \frac{\eta(k+1)}{\tau(k) k_{s2v} k_{scr}(k) u_1(k)}
        = \frac{\eta(k)}{\tau(k-1) k_{s2v} k_{scr}(k-1) u_1(k-1)} \times \gamma_{proc}(k-1)\\
                        \\ + \Gamma(k-1) t_s k_{ads}(k-1) \con{NH_3}^{in}(k-1)
\end{multline*}
Thus, we have the recursive equation for change in concentration due to reduction:
\begin{multline}
        \eta(k+1) = \eta(k) \times \lr{\frac{\tau(k)}{\tau(k-1)}}
                                \times \lr{\frac{u_1(k)}{u_1(k-1)}}
                                \times \lr{\frac{k_{scr}(k)}{k_{scr}(k-1)}}
                                \times \gamma_{proc}(k-1)
                                \\
                        + t_s k_{s2v} \times \lrf{\con{NH_3}^{in} \Gamma(k-1) \tau(k) u_1(k)} \times \lr{k_{scr}(k) k_{ads}(k-1)}
\end{multline}
Explicitly writing the individual terms:
\begin{align*}
        \eta(k+1) =& \eta(k) \lrb{\frac{\tau(k)}{\tau(k-1)}}
                                \lrb{\frac{u_1(k)}{u_1(k-1)}}
                                \lrb{\frac{k_{scr}(k)}{k_{scr}(k-1)}} \\
                &-\eta(k) \lrb{\frac{\tau(k)}{\tau(k-1)}}
                                \lrb{\frac{u_1(k)}{u_1(k-1)}}
                                \lrb{\frac{k_{scr}(k)}{k_{scr}(k-1)}}
                t_s k_{ads}(k-1) \con{NH_3}^{in}(t-1)
                \\
                &-\eta(k) \lrb{\frac{\tau(k)}{\tau(k-1)}}
                                \lrb{\frac{u_1(k)}{u_1(k-1)}}
                                \lrb{\frac{k_{scr}(k)}{k_{scr}(k-1)}}
                t_s k_{scr}(k-1) u_1(k-1)
                \\
                &-\eta(k) \lrb{\frac{\tau(k)}{\tau(k-1)}}
                                \lrb{\frac{u_1(k)}{u_1(k-1)}}
                                \lrb{\frac{k_{scr}(k)}{k_{scr}(k-1)}}
                t_s k_{od}(k-1)
                \\
                &+ t_s k_{s2v} \times \lrf{\Gamma(k-1) \tau(k) u_1(k) \con{NH_3}^{in}} \times \lr{k_{scr}(k) k_{ads}(k-1)}
\end{align*}
The above equation can be simplified using the following two assumptions:
\begin{itemize}
        \item[$A1.$] The temperature doesn't change significantly across contiguous samples, i.e., $T(k-1) \approx T(k)$.
        \begin{align}
                \implies \frac{k_{scr}(k)}{k_{scr}(k-1)} \approx 1
        \end{align}
        \item[$A2.$] The product of rate constants results in a combined rate-constant,
        \begin{align}
                k_{scr}(k) k_{ads}(k-1) &\approx k_{scr}(k-1) k_{ads}(k-1)  = A_{scr}A_{ads} \exp\lrf{\frac{E_{scr}+E_{ads}}{R T(k-1)}} = k_{scr/ads}(k-1)
        \end{align}
\end{itemize}
Incorporating the above assumptions we have the dynamic model for change in concentration due $NO_x$ reduction:
\begin{align*}
         \eta(k+1) =& \eta(k) \lrb{\frac{\tau(k)}{\tau(k-1)}}
                                \lrb{\frac{u_1(k)}{u_1(k-1)}}
                \\
                &-\eta(k) \lrb{\frac{\tau(k)}{\tau(k-1)}}
                                \lrb{\frac{u_1(k)}{u_1(k-1)}}
                t_s k_{ads}(k-1) \con{NH_3}^{in}(t-1)
                \\
                &-\eta(k) \lrb{\frac{\tau(k)}{\tau(k-1)}}
                                \lrb{\frac{u_1(k)}{u_1(k-1)}}
                t_s k_{od}(k-1)
                \\
                &-\eta(k) \lrb{\frac{\tau(k)}{\tau(k-1)}}
                t_s k_{scr}(k) u_1(k)
                \\
                &+ t_s k_{s2v} \times \lrf{\Gamma(k-1) \tau(k) u_1(k)\con{NH_3}{in}} \times k_{scr/ads}(k-1)
\end{align*}
\begin{equation}
       \label{eqn::nox_govern}
\end{equation}
