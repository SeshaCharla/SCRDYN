\subsection{Hybrid model for wider temperature ranges}

The only restrictive assumption in the derivation of the above model are related to the temperature range of operation.
The linear temperature dependence of rate constants and independence of density (thereby residence time) and void
fraction of the catalyst are valid for a narrow range of temperatures. The other nonlinear effects due to flow-rate and
urea injection are captured in the model. Introducing more complex temperature models for the physical quantities would
lead to parametric models that are not linear (in parameters) and identifiable. Thus, a hybrid model is proposed where
the model structure remains the same while the parameters switch depending on the temperature.

\begin{multline}
        x(k+1) = u_1(k) - \eta(k) \lrb{\frac{u_1(k)}{F(k)}} \lrb{\frac{F(k-1)}{u_1(k-1)}}
                        + \lrb{\frac{u_1(k)}{F(k)}} \pmb \phi^{T}_{NO_x}(k) \pmb \theta^i_{NO_x}
        \\
        T(k), T(k-1) \in \lrb{T_i-\Delta T, T_i + \Delta T}
\end{multline}

Thus, the system switches between models when the current temperature is beyond the $\pm \Delta T$ range of the model.
Experience form the linearized model identification suggest that, $\Delta T \approx 50 \, ^0C$ would account for
the changes in systems parameters due to drastic changes in temperatures.

As the model doesn't have explicit dependence on the unmeasured initial conditions, the parameters for the switched system can be identified by partitioning the data based on the temperature.

\begin{align}
        e.g., \quad T \in \lrf{100, 200, 300, 400} \; ^0C
\end{align}
