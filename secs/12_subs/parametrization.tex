\subsection{Parametrizing the $\eta$ dynamics}
The equation (\ref{eqn::nox_govern}) can be written in the following structure, with $g_i$'s denoting the corresponding expressions in each of the terms.
\begin{align}
        \eta(k+1) &= \eta(k) \lrf{ g1 - g2 -g3 -g4} + g0
\end{align}
The individual terms, are parametrized based on following relevant set of assumptions:
\begin{itemize}
        \item[$A3.$] The rate constant is linear for a given operating range of temperature.
        \begin{align}
                k_i &= m_i T + c_i = \underbrace{\bm{T & 1}}_{\pmb \phi^T} \underbrace{\bm{m_i \\ c_i}}_{\pmb \theta_i}
                        = \pmb \phi^{T} \pmb \theta_i
                \qquad \forall \quad T \in \lrb{T_0, T_0 + \Delta T}
                \label{eqn::k_mdl}
        \end{align}
        %===
        \item[$A4.$] $\Gamma$ is a constant for a given operating range of temperature and only changes with aging.
        \begin{align}
                \Gamma - \text{constant} \quad \forall T \in \lrb{T_0, T_0 + \Delta T}
                \label{eqn::gamma_mdl}
        \end{align}
        %===
        \item[$A5.$] The model for $\con{NH_3}^{in}$ based on urea injection is given by equation (\ref{eqn::urea_parm}), i.e., $\con{NH_3}^{in}$ depends only on the flow-rate and urea injection but not the temperature (as the urea is preheated).
        \begin{align}
                \con{NH_3}^{in}(k) &= \nu_u \times \frac{u_2(k)}{F(k)}
                \label{eqn::urea_mdl}
        \end{align}
        \item[$A6.$] The model for residence time is given by equation (\ref{eqn::res_time}), i.e., the residence time depends only on the flow-rate and effect of change in density (due to change in temperature) is negligable.
        \begin{align}
                \tau(k) &= \frac{V \rho_0}{F(k)} = \frac{\tau_0}{F(k)}
                \label{eqn::residence_time_mdl}
        \end{align}
\end{itemize}

Using the above assumptions, we have the parametrization of individual terms
\begin{enumerate}
%====
\item \begin{align*}
        g_0 &= t_s k_{s2v} \times \lrf{\Gamma(k-1) \tau(k) u_1(k)} \times k_{scr/ads}(k-1)\\
                &= t_s k_{s2v} \Gamma \times \frac{\tau_0}{F(k)} \times u_1 (k) \times \pmb \phi^T(k-1) \pmb \theta_{scr/ads}
                \qquad \lrb{\because \ref{eqn::gamma_mdl}, \ref{eqn::residence_time_mdl}, \ref{eqn::k_mdl}}
\end{align*}
\begin{align}
        g_0 &= \lrf{ \frac{u_1(k)}{F(k)} \pmb \phi(k-1) } \times \lrf{ t_s k_{s2v} \Gamma \tau_0 \pmb \theta_{scr/ads} }
\end{align}

\end{enumerate}
