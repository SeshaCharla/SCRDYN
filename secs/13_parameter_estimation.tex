\newpage
\section{Parameter estimation}
The process model for $NO_x$ input and output dynamics is put into the regression form in equation
\ref{eqn::ident_form}. Least-squares cannot be directly applied to the above regression model as the condition number of
the resulting matrix is high. This is due to the fact that individual elements of the $\pmb \phi_{NO_x}$ vector are of
significantly different order of magnitudes. This numerical issue can be resolved by scaling the equation appropriately.
\begin{align*}
        y_{NO_x} &= \pmb \phi_{NO_x}^T \pmb \theta_{NO_x} + \varepsilon\\
        \text{Let, } \qquad W &= diag \bm{w_1, w_2, \hdots, w_8}, \quad w_y \\
        \text{then, } \quad w_y y_{NO_x} &= \underbrace{\pmb \phi_{NO_x}^T  W}_{\pmb \phi_{wNO_x}}
                        \underbrace{ W^{-1} \pmb \theta_{NO_x} w_y }_{\pmb \theta_{wNO_x}} + w_y \varepsilon\\
        \implies y_{w NO_x} &= \pmb \phi_{wNO_x} \pmb \theta_{w NO_x} + \varepsilon_w
\end{align*}
Thus, scaling the regressor in above fashion doesn't affect the model structure errors. The actual parameter vector $\pmb \theta_{NO_x}$ can be obtained by multiplying the scaling matrix to the estimate again.
\begin{align*}
        \hat{\pmb \theta}_{NO_x} &=  \frac{1}{w_y} W \hat{\pmb \theta}_{wNO_x}
\end{align*}
The elements of W are chosen such that the order of magnitude of each of the elements of $\pmb \phi_r$ are same as that of $\pmb y_{w NO_x}$.

From the available test data, we have the elements of scaling matrix:
\begin{table}[H]
        \centering
        \begin{tabular}{c c c c c c c c c c c}
                $w_0$  & $w_1$  & $w_2$     & $w_3$     & $w_4$     & $w_5$     & $w_6$  & $w_7$ & $w_y$\\
                $1e-1$ & $1$    & $1e-2$    & $1e-1$    & $1e-3$    & $1e-2$    & $1$    & $10$  & $10$
        \end{tabular}
\end{table}

These weights are used for both parameter estimation using the least-squares and the recursive least-squares.
