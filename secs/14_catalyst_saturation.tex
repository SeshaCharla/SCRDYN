\newpage
\section{Catalyst Saturation}

The model developed till now considers the situation where the urea-dosing is such that the catalyst neither saturated
nor empty. The ideal case which is also an assumption in the linearized CSTR model. The dynamic model for catalyst's
surface concentration, when considering the saturation and zero concentration limit will be ($\gamma_{proc}(k)$ is
defined in \ref{eqn::gamma_proc}):
%=====
\begin{align}
        \sigma^{ub}(k) &= \sigma (k-1) \gamma_{proc}(k-1) + t_s \Gamma k_{ads}(k-1) \con{NH_3}^{in}(k-1)
        \label{eqn::unbounded_sigma_mdl}\\
        \sigma(k) &= \begin{cases}
                                \sigma^{ub}(k) & \text{if }  0 \leq \sigma^{ub}(k) \leq \Gamma\\
                                \Gamma         & \text{if }  \sigma^{ub}(k) > \Gamma \\
                                0              & \text{if }  \sigma^{ub}(k) < 0
                        \end{cases}
        \label{eqn::actual_sigma_mdl}
\end{align}

Thus substituting, equation \ref{eqn::sigma_elim} into above equations \ref{eqn::unbounded_sigma_mdl} and \ref{eqn::actual_sigma_mdl}:
%===
\begin{align}
        \sigma^{ub}(k) &= \lrf{\frac{\eta(k)}{\tau(k-1) k_{s2v} k_{scr}(k-1) u_1(k-1)}} \gamma_{proc}(k-1) + t_s \Gamma k_{ads}(k-1) \con{NH_3}^{in}(k-1)\\
        % ====
        \frac{\eta(k+1)}{\tau(k) k_{s2v} k_{scr}(k) u_1(k)} &=
        \begin{cases}
                                \sigma^{ub}(k)     & \text{if }  0 \leq \sigma^{ub}(k) \leq \Gamma\\
                                \Gamma             & \text{if }  \sigma^{ub}(k) > \Gamma \\
                                0                  & \text{if }  \sigma^{ub}(k) < 0
        \end{cases} \\
        % =====
        \implies \eta(k+1) &=
        \begin{cases}
        \lrf{\tau(k) k_{s2v} k_{scr}(k) u_1(k)}\sigma^{ub}(k)         & \text{if }
                0 \leq \sigma^{ub}(k) \leq \Gamma\\
        \lrf{\tau(k) k_{s2v} k_{scr}(k) u_1(k)} \Gamma                & \text{if }
                \sigma^{ub}(k) > \Gamma \\
        0                                                             & \text{if }
                \sigma^{ub}(k) < 0
        \end{cases}
        \label{eqn::eta_cases}
\end{align}
%====
From, the derivation and the parametrization in the last section, specifically, equations \ref{eqn::eta_parm} and
\ref{eqn::phi_NOx}:
\begin{align}
        \lrf{\tau(k) k_{s2v} k_{scr}(k) u_1(k)}\sigma^{ub}(k) &=
        \eta(k) \lrb{\frac{u_1(k)}{F(k)}} \lrb{\frac{F(k-1)}{u_1(k-1)}}
        +\lrb{\frac{u_1(k)}{F(k)}} \pmb \phi_{NO_x}^T \pmb \theta_{NO_x}
        \label{eqn::sigma_ub_param}
\end{align}
%===
Similarly, the second case can be parametrized using the same set of approximations (Ai's) and models of physical quantities,
\begin{align*}
        \lrf{\tau(k) k_{s2v} k_{scr}(k) u_1(k)} \Gamma  &= \frac{\tau_0}{F(k)} \times k_{s2v} \times \pmb \phi(k)^T \pmb \theta_{scr} \times u_1(k) \times \Gamma\\
        % ==
        &= \lrf{\lrb{\frac{u_1(k)}{F(k)}} \pmb \phi^T(k) } \lrf{\Gamma \tau_0 k_{s2v} \pmb \theta_{scr}}
\end{align*}
\begin{align}
        Let, \qquad  \Gamma \tau_0 k_{s2v} \pmb \theta_{scr} &= \pmb \theta_{\Gamma scr}\\
        \lrf{\tau(k) k_{s2v} k_{scr}(k) u_1(k)} \Gamma  &=
        \lrb{\frac{u_1(k)}{F(k)}} \pmb \phi^T(k)  \pmb \theta_{\Gamma scr}
        \label{eqn::gamma_scr}
\end{align}
%===
The conditions for case switching can also be parametrized to the same expressions as (\ref{eqn::sigma_ub_param}) and (\ref{eqn::gamma_scr}) by multiplying both sides of the inequalities by $\lrf{\tau(k) k_{s2v} k_{scr}(k) u_1(k)} (>0)$ which is always positive.

Let,
\begin{align}
        f_{\sigma}(k) &= \eta(k) \lrb{\frac{u_1(k)}{F(k)}} \lrb{\frac{F(k-1)}{u_1(k-1)}}
        +\lrb{\frac{u_1(k)}{F(k)}} \pmb \phi_{NO_x}^T \pmb \theta_{NO_x}\\
        %===
        f_{\Gamma}(k) &= \lrb{\frac{u_1(k)}{F(k)}} \pmb \phi^T(k)  \pmb \theta_{\Gamma scr}
\end{align}

Substituting the above parametrization into equation (\ref{eqn::eta_cases}):
%===
\begin{align}
        \eta(k+1) &=
        \begin{cases}
                f_{\sigma}(k) & \text{if } 0 \leq f_{\sigma}(k) \leq f_{\Gamma}(k)\\
                f_{\Gamma}(k) & \text{if } f_{\sigma}(k) > f_{\Gamma}(k) \\
                0             & \text{if } f_{\sigma}(k) < 0
        \end{cases}
\end{align}
The above cases can be written is min-max form as follows:
\begin{align}
        \eta\lr{k + 1} &= \max \lrf{ 0, \min \lrf{f_{\sigma}(k), f_{\Gamma}(k)}}
\end{align}
%===
Writing the same in terms of the tail-pipe $NO_x$ concentration,
\begin{align}
        x_1(k+1) &= u_1 -
        \begin{cases}
                f_{\sigma}(k) & \text{if } 0 \leq f_{\sigma}(k) \leq f_{\Gamma}(k)\\
                f_{\Gamma}(k) & \text{if } f_{\sigma}(k) > f_{\Gamma}(k) \\
                0             & \text{if } f_{\sigma}(k) < 0
        \end{cases}
        \\
        x_1(k+1) &= u_1(k) - \max \lrf{ 0, \min \lrf{f_{\sigma}(k), f_{\Gamma}(k)}}
\end{align}
%===
Thus, the tailpipe $NO_x$ dynamics are bounded by:
\begin{align}
        u_1(k) - f_{\Gamma}(k) \leq x_1(k+1) \leq u_1(k)
\end{align}
%====
Or, the $NO_x$ reduction dynamics are bounded by:
\begin{align}
        0 \leq \eta(k+1) \leq f(\Gamma)(k)
\end{align}
% ====
This relationship can be used to find the optimal values of the parameter $\pmb \theta_{\Gamma scr}$ by solving the following constrained optimization problem:
\begin{align*}
        \text{Minimize:}& \qquad  \norm{\pmb \theta_{\Gamma scr}} \\
        \text{Subject to:}& \qquad \eta(k+1) \leq f_\Gamma(k) \qquad \forall \: k \\
        \text{where, } & \qquad f_{\Gamma}(k) = \lrb{\frac{u_1(k)}{F(k)}} \pmb \phi^T(k)  \pmb \theta_{\Gamma scr}
\end{align*}
\begin{align}
        \label{eqn::optimization_prob}
\end{align}
