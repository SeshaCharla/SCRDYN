\subsection{Saturated Catalyst Model Parameter Estimation}

The linear programming problem is solved using a quadratic polynomial approximation for $\pmb \phi \pmb \theta_{\Gamma scr}$. Both, $k_{scr/asc}$ and $\Gamma$ are assumed to be monotonic functions of temperature but in opposing directions. Thus, just a linear model for the product will not capture the change in monotonic of the product that happens at a particular temperature. After a systematic analysis of various polynomial orders and temperature partitions, a quadratic model with two partitions was found to be the best approximation with minimum prediction error. The parameter estimates of the saturated model are tabulated below.

\begin{table}[H]
        \centering
        \begin{tabular}{l l c c c c}
                \hline \hline
                Age & Test & Temp. Zone &
                $\pmb \theta_{\Gamma scr}[2]$ &
                $\pmb \theta_{\Gamma scr}[1]$ &
                $\pmb \theta_{\Gamma scr}[0]$ \\ \hline \hline
                % ============================================
                Degreened & RMC & high & & & \\
                % =============================================
                Aged & RMC & high & & & \\
                % =============================================
                Degreened & hot-FTP & high & & & \\
                % =============================================
                Aged & hot-FTP & high & & & \\
                % =============================================
                Degreened & cold-FTP & high & & & \\
                % =============================================
                Aged & cold-FTP & high & & & \\
                % =============================================
                Degreened & cold-FTP & low & & & \\
                % =============================================
                Aged & cold-FTP & low & & & \\
                % =============================================
                \hline \hline
                % =============================================
        \end{tabular}
\end{table}
