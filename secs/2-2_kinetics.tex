\subsection{General reaction kinetics}
For a stoichiometric reaction of the following form:
\begin{align*}
    a A + b B \longrightarrow c C + d D
\end{align*}
We have the reaction rate \cite{chem_kine}:
\begin{align}
    r &= -\frac{1}{a} \frac{d [A]}{dt}
       = -\frac{1}{b} \frac{d [B]}{dt}
       = \frac{1}{c}  \frac{d [C]}{dt}
       = \frac{1}{d}  \frac{d [D]}{dt}
       \label{eqn::react_rate}\\
    r &= k [A]^m [B]^n
        \label{eqn::rate}
\end{align}
Where,
\begin{align*}
    [\bullet] &- \text{Concentration of the reactant } \bullet\\
    k &- \text{Rate constant}\\
    m, n &- \text{Constant exponents, ($m+n$) is the order of reaction}\\
\end{align*}
In the subsequent derivations, the exponents in the rate equations are limited
to either zero or 1 $m, n \in \{0, 1\}$ (This is consistent with the assumption
that reaction rates are proportional to gas-phase concentrations from \cite{devarakonda2009model}).

The rate constant can be determined using \itbf{Arrhenius equation}:
\begin{align}
    k &= A e^{E_a/RT}
    \label{eqn::arrh}
\end{align}
\begin{align*}
    \text{Where,} \quad &\\
    A &- \text{Pre-exponential factor}\\
    E_a &- \text{Activation energy}\\
    T &- \text{Temperature}\\
    R &- \text{Universal gas constant}
\end{align*}
\subsubsection{Effects of temperature change on rate constant}
We have the Arrhenius equation:
\begin{align}
    k &= A e^{-E_a/RT}
\end{align}
\begin{align*}
    \implies \delta k &= \delta T A \lr{\frac{E_a}{RT^2}} e^{-E_a/RT} = \delta T k \lr{\frac{E_a}{RT^2}}
\end{align*}
In general, $E_a$ is several orders of magnitude greater than the other
parameters. Thus, small change in temperature gets amplified as the change in
the rate-constant.

Also, we have the first-order tayler approximation for the rate-constant
variation with temperature:
\begin{align*}
    k(T_0 + \delta T) &\approx \bar k(T_0) + \frac{E_a}{RT_0^2} \bar k(T_0) \delta T
\end{align*}
\begin{align}
    k(\delta T) &\approx \bar k + p \bar k \delta T
\end{align}
Where,
\begin{align*}
    p &= \frac{E_a}{RT_0^2}
\end{align*}
This form would be more amenable to parameter estimation despite introducing
approximation errors.



%===============================================================================
\subsection{Eley-Rideal Mechanism}
In this mechanism, proposed in 1938 by D. D. Eley and E. K. Rideal, only one of
the molecules adsorbs and the other one reacts with it directly from the gas
phase, without adsorbing ("non-thermal surface reaction") \cite{eley_rideal}.

\itbf{Note:} The concentration $(\lrb{\bullet})$ can mean either surface
concentration or volumetric concentration based on the context. The
rate-constants take care of the necessary unit conversions.

\begin{align*}
    A^{g} + \Theta_{free} &\rightleftharpoons A^{ads}  \qquad (k_1, k_{-1})\\
    A^{ads} + B^{g} &\longrightarrow Products \qquad (k)
\end{align*}

We have the rate of the second reaction:
\begin{align*}
    r &= k \con{A^{ads}} \con{B}
\end{align*}
Thus,
\begin{align*}
    \frac{d \con{A^{ads}}}{dt} &= k_1 [A] \con{\Theta - \mol{A^{ads}}} - k_{-1} \con{A^{ads}} - k \con{A^{ads}} \lrb{B}
\end{align*}

\itbf{Note:} If the volume of the reaction chamber is constant. The
concentration $\con{.}$ can be replaced with the number of moles $\mol{.}$ without changing the structure of the equations. Only the values of the rate
constants change accommodating the change in units and the volume/area when required.

Thus,
\begin{align}
    \frac{d \mol{A^{ads}}}{dt} &= k_1 [A] \lr{\Theta - \mol{A^{ads}}}
                                - k_{-1} \mol{A^{ads}}
                                - k \mol{A^{ads}} \con{B}
    \label{eqn::eley_rideal}
\end{align}

Where,
\begin{align*}
    \Theta &- \text{Molar storage capacity of the catalyst:}\\
           &\qquad \text{The total number of moles that can be adsorbed on to the surface}\\
    \lrb{\bullet} &= \text{Concentration of $\bullet$}\\
    \lrf{\bullet} &= \text{Number of moles of $\bullet$}
\end{align*}

%
