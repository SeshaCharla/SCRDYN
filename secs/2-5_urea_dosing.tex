\section{Ammonia input (Urea Dosing) dynamics}
The actual input to the system is urea [from AdBlue ($32.5\%$ aqueous urea
solution)] injection that converted to ammonia (through reactions:
(\ref{eqn::urea_1}), (\ref{eqn::urea_2}) and (\ref{eqn::urea_3})). This can be modelled by the following equation \cite{nova2014urea}:
\begin{align*}
    \dot C_{NH_3, in} &= - \frac{1}{\tau} C_{NH_3, in} + 2 \frac{1}{\tau} \frac{ \eta u_{inj}}{N_{urea} F}\\
    \text{where, } \quad &\\
    \tau &- \text{Time constant}\\
    u_{inj} &- \text{Mass injection rate of the AdBlue solution}\\
    \eta &- \text{Mass fraction of urea in the solution}\\
    N_{urea} &- \text{Atomic number of urea}\\
    F &- \text{Exhaust flow rate of the catalyst } m^3/s
\end{align*}

\itbf{Assumptions}:
\begin{enumerate}
    \item The above model assumes that the evaporation of the urea-solutions
        (reaction: \ref{eqn::urea_1}) is a significantly slower process as
        compared to it's decomposition into ammonia. Thus, the reaction
        kinetics are neglected and the eveporation is considerd are a first
        order process w.r.t the vapour pressure of the ammonia.
    \item The injection dynamics are completely decoupled from that of other
        states.
    \item Further, it is observed that Urea is completely converted to Ammonia
        at the very upstream part of the SCR catalyst
        \cite{hsieh2011development}.
\end{enumerate}

Reparametrizing the above equation, let,
\begin{align*}
    x_4 &= C_{NH_3, in} \qquad b_u = 2 \frac{ \eta}{N_{urea}} \qquad \omega_u = \frac{1}{\tau}\\
    u_2 &= u_{inj}
\end{align*}

\begin{equation}{\label{eqn::urea_inj}}
    \dot x_4 = - \omega_u x_4 +   \frac{\omega_u b_u}{F} u_{2}
\end{equation}
