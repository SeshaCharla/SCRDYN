\newpage
\section{SCR/ASC Reactions}
The following lists all the reactions that take place inside the SCR-ASC chamber.

\begin{align*}
    NH_2 - CO - NH_2 (liquid) &\longrightarrow NH_2 - CO - NH_2^* + x H_2 O
                & &[\text{AdBlue evaporation}] \\
    NH_2 - CO - NH_2^*  &\longrightarrow  HNCO + NH_3
                & &[\text{Urea decomposition}] \\
    HNCO + H_2O &\longrightarrow NH_3 + CO_2
                & &[\text{Isocynic acid hydrolysis}] \\
    %===
    NH_3 + \Theta_{free} &\leftrightharpoons NH_3(ads)
                         & &[\text{Ammonia Adsorption/Desorption}]\\
    %===
    4 NH_3 (ads) + 4 NO + O_2 &\longrightarrow 4 N_2 + 6 H_2O
                              & &[\text{Standard SCR reaction}]\\
    %===
    2 NH_3 (ads) +  NO + N O_2 &\longrightarrow 2 N_2 + 3 H_2O
                              & &[\text{Fast SCR reaction}]\\
    %===
    4 NH_3 (ads) + 3N O_2 &\longrightarrow 3.5 N_2 + 6 H_2O
                              & &[\text{Slow SCR reaction}]\\
    %===
    4 NH_3 + 3 O_2 &\longrightarrow 2 N_2 + 6 H_2O
                         & &[\text{AMOX with/without ASC}]\\
    4 NH_3 + 5 O_2 &\longrightarrow 4 NO + 6 H_2 O
                         & &[\text{AMOX with/without ASC}]\\
    2 NH_3 + 2 O_2 &\longrightarrow N_2O + 3 H_2O
                         & &[\text{AMOX with/without ASC}]\\
    %==
    2 NO + O_2 &\longrightarrow 2 NO_2
                        & &[\text{NO oxidation}]
\end{align*}

The Eley-Rideal reaction mechanism \cite{yuan2015diesel}, \cite{hsieh2011development}, \cite{nova2014urea} is considered
for interpreting the SCR reactions, where one reactant $(NO_x)$ is gaseous, and the other is adsorbed on the catalyst
surface $(NH_3)$.

Further, in order to keep the model order reasonably low, only the following three reactions are considered:
\begin{enumerate}
    \item Standard SCR reaction:
    \begin{align}
        4 NH_3 ^{ads} + 4 NO + O_2 &\xrightarrow[]{k_{scr}} 4 N_2 + 6 H_2O \label{eqn::std_scr}
    \end{align}
    \item Ammonia Oxidation:
    \begin{align}
        4 NH_3^{ads} + 3 O_2 &\xrightarrow[]{k_{oxi}} 2 N_2 + 6 H_2O \label{eqn::amox}
    \end{align}
    \item Ammonia Adsorption/Desorption:
        \begin{align}
            NH_3 + \Theta_{free} &\xrightleftharpoons[k_{des}]{k_{ads}} NH_3^{ads}
            \label{eqn::ads}
        \end{align}
\end{enumerate}


\subsection{Temperature model for rate constants}
The rate constants $(k)$ of the reactions are temperature dependent and follow the Arrhenius equation:
\begin{align*}
    k = A \exp\left(-\frac{E}{RT}\right)
\end{align*}
where $A$ is the pre-exponential factor, $E$ is the activation energy, and $R$ is the universal gas constant. The small
perturbation form of the above equation is given by:
\begin{align*}
    \delta k &= A e^{-\frac{E}{RT}} \underbrace{\lr{\frac{E}{RT^2}}}_p \delta T = k p \delta T
\end{align*}
Thus,
\begin{align*}
    k(T) \approx k(T_0) + \delta k = k(T_0) + k(T_0) p(T_0) \underbrace{\lr{T - T_0}}_{\delta T}
\end{align*}

Hence, within a certain range of temperatures, the rate constant can be assumed to be varying linearly with temperature.
Thus, the rate constant can be:
\begin{align*}
    k(T) &= mT + c \qquad  \text{for } \: T \in [T_0 - \Delta T_{max}, T_0 + \Delta T_{max}]
\end{align*}

Based on the available data, $T_0$ is chosen as $250 \lx{^o}{C}$ and $\Delta T_{max}$ such that it spans all the
available data. From linear model fitting, the validity of the linear model was found to be limited to $\pm 50
\lx{^o}{C}$.

\subsubsection{Quadratic temperature model for rate constants}
Using second-order Taylor series approximation, the expression for rate constant can be rewritten as:
\begin{align*}
    k(T) &\approx k(T_0) + k'(T_0) (T - T_0) + \frac{1}{2} k''(T_0) (T - T_0)^2\\
    \text{Let,} \qquad
    &m + 2qT_0 = k'(T_0) = \lr{\frac{AE}{RT_0^2}} e^{-\frac{E}{RT_0}} \\
    &q = k''(T_0) = \lr{\frac{A E^2}{R^2 T_0^4} - \frac{A E}{2 R T_0^3}} e^{-\frac{E}{RT_0}} \\
    \implies k(T) &\approx k(T_0) + \lr{m + 2qT_0} (T - T_0) + q (T - T_0)^2
                   = q T^2 + m T + \underbrace{\lr{ -  q T_0^2 - mT_0  + k(T_0)}}_c
\end{align*}
Thus, we can have a quadratic approximation model for the rate constant:
\begin{align}
    k(T) &= q T^2 + m T + c \qquad \text{for } \: T \in [T_0 - \Delta T_{max}, T_0 + \Delta T_{max}]
\end{align}

The subsequent derivation will use the quadratic model for the rate constant and this can be switched to linear model by setting $q = 0$.


%===============================================================================
\subsection{Model reduction: Lumping ASC reaction dynamics into SCR reaction dynamics}

\begin{figure}[H]
    \centering
    \includegraphics[width=0.9\textwidth]{./figs/scr_sys/SCR-ASC_ModelReduction_horizontal.png}
    \caption{Schematic of the SCR-ASC reduced model}
    \label{fig:scr-asc}
\end{figure}

The ammonia oxidation (AMOX) reactions are similar on SCR and ASC catalysts. First, the gaseous ammonia adsorbs onto the
catalyst system. Then, the adsorbed ammonia oxidizes into $N_2$ or $NO_x$ based on the temperature and other conditions.
As the AMOX reactions on both the catalysts are similar, it is not possible to distinguish the origin of the products
from the outlet measurements alone unless measurements between SCR and ASC sections are available.

Thus, oxidation of adsorbed ammonia on both SCR and ASC catalysts can be  combined into a single reaction, lumping the
rates constants and concentrations of the products into parameters and states of a single oxidation reaction. Further
the nitrogen selectivity of that single AMOX reaction is assumed to be $100\%$. This is valid for temperatures greater
than $225 \lx{^o}{C}$ \cite{jain2023diagnostics}, i.e., the temperature range of interest for most of the test and road
conditions.

This aggregation of reactions results in errors in the rate constant estimates, of all the SCR-ASC reactions.
Specifically, the rate constant estimate for the $NO_x$ reduction will be lower than the actual value as only a part of
the total adsorbed ammonia (on SCR catalyst) alone is involved in the reduction reaction.
