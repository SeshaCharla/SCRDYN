\section{Mean Residence Time}

\itbf{Definition}: Mean residence $(\tau)$ time is the average time a parcel of
fluid spends inside the reactor. It is defined as the ratio of the volume of
the reactor to the volumetric flow rate of the fluid.

\begin{align}
    \tau = \frac{V}{f_v}
\end{align}

The actual residence time of the fluid inside the reactor follows a distributing
based on the type of the reacting flow (PFR, CSTR, etc) with mean as the mean
residence time $(\tau)$.

For the given test-cell and truck data, the mean residence time is calculated
with the following values:

\begin{align*}
    \text{Length of the chamber:}&&
    L &= L_{scr} + L_{asc} = 9.5 \, in  + 2 \, in = 11.5 \, in = 29.21 \, cm\\
    \text{Diameter of the chamber:} &&
    D &= 13 \, in = 33.02 \, cm\\
    \text{Volume of the chamber:} &&
    V &= \frac{\pi}{4} D^2 L = \frac{\pi}{4} \times 33.02^2 \times 29.21 = 25013.543 \, cm^3\\
    \text{Nominal Density:} &&
    \rho(250 \lx{^o}{C}) &= 6.75e-4 \, g/ml\\
    \text{Nominal Mass Flow rate:} &&
    F &= 196 \, g/s\\
\end{align*}

Thus, we have the mean residence time as:

\begin{align}
    \tau &= \frac{V}{f_v} = \frac{\rho V}{F} = \frac{6.75e-4 \times 25013.543}{196} = 0.086 \, s
\end{align}

Detailed plots of residence time calculations for the test-cell and truck data
are presented in appendix-\ref{app:res_time_plots}.
% ==============================================================================

The above calculations show that the mean residence time is around $0.1 \, s$
that is half that of the sampling time of the test-cell data (0.2 s) and
one-tenth of the sampling time($\Delta t$) of the truck data (1 s). This
implies that the measurement signal of the gas concentrations don't capture the
reaction transients that generally occur at the time scales  that are less than
the mean residence time. This also prompts us to develop an "averaged"
nonlinear ARMAX model for the system that captures the dynamics of the system
at the time scales of the sampling time while capturing the integrating (and/or
memory) effects of the catalyst storage at the end of each residence time
within the sample.

% ==============================================================================

\begin{table}[H]
\caption{Mean residence time of individual data sets}
\begin{minipage}{0.49\textwidth}
\begin{table}[H]
    \centering
    \begin{tabular}{r l l}
        \hline
        \hline
        \textbf{Test-cell Data} & \textbf{$\tau$ (s)} & \textbf{$\Delta t$ (s)}\\
        \hline
        \hline
        $dg\_rmc$ & 0.07 & 0.2\\
        $dg\_hftp$ & 0.16 & 0.2\\
        $dg\_cftp$ & 0.16 & 0.2\\
        $aged\_rmc$ & 0.07 & 0.2\\
        $aged\_hftp$ & 0.16 & 0.2\\
        $aged\_cftp$ & 0.16 & 0.2\\
        \hline
        \hline
    \end{tabular}
\end{table}
\end{minipage}
\begin{minipage}{0.49\textwidth}
\begin{table}[H]
    \centering
    \begin{tabular}{r l l}
        \hline
        \hline
        \textbf{Truck Data} & \textbf{$\tau$ (s)} & \textbf{$\Delta t$ (s)}\\
        \hline
        \hline
        $adt\_15$ & 0.08 & 1\\
        $adt\_17$ & 0.06 & 1\\
        $mes\_15$ & 0.1 & 1\\
        $mes\_18$ & 0.08 & 1\\
        $trw\_15$ & 0.09 & 1\\
        $trw\_18$ & 0.08 & 1\\
        $wer\_15$ & 0.08 & 1\\
        $wer\_17$ & 0.09 & 1\\
        \hline
        \hline
    \end{tabular}
\end{table}
\end{minipage}
\end{table}
