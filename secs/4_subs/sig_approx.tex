\subsection{Approximation of $\sigma(k)$}
\subsubsection{Approximation of $\sigma(k)$ as $NH_3^{ads}(k)$ (Zero-order hold)}
\itbf{Note:} $\sigma(k)$ can be approximated as the surface concentration at the beginning of the sample time.

The average surface concentration $\sigma(k)$ in the sample time is approximated
as the surface concentration at the beginning of the sample time. This
approximation will introduce an error in the model whose effects will be
analyzed in sequential sections.

Thus,
\begin{align}
    \Omega(k) &\approx n \tau \lr{\Gamma \gamma_{ads} (k) - \con{NH_3}^{ads}(k) \gamma_{des} (k)}
\end{align}

Thus, we have approximate ammonia adsorption/desorption process dynamics model as:

\begin{align*}
    \con{NH_3}^{ads}(k + 1) &= \con{NH_3}^{ads} (k) + n \tau \lr{\Gamma \gamma_{ads} (k) - \con{NH_3}^{ads}(k) \gamma_{des} (k)}\\
    &= n \tau \Gamma \gamma_{ads} (k) + \con{NH_3}^{ads} (k) \lr{1 - n \tau \gamma_{des} (k)}\\
    &= n \tau \Gamma k_{ads} \con{NH_3}^{in}(k)  + \con{NH_3}^{ads} (k) \lr{1 - n \tau \lr{k_{ads} \con{NH_3}^{in}(k) + k_{des} + k_{scr} \con{NO_x}^{in}(k) + k_{oxi}}}
\end{align*}


\begin{align}
    \con{NH_3}^{ads}(k + 1) &= n \tau \Gamma k_{ads} \con{NH_3}^{in}(k)  + \con{NH_3}^{ads} (k) \lr{1 - n \tau \lr{k_{ads} \con{NH_3}^{in}(k) + k_{des} + k_{scr} \con{NO_x}^{in}(k) + k_{oxi}}}
\end{align}

Examining the individual terms:
\begin{align*}
    \con{NH_3}^{ads}(k + 1) =& \con{NH_3}^{ads}(k)\\
        &+ n\tau k_{ads} \con{NH_3}^{in} \lr{\Gamma - \con{NH_3}^{ads}(k)} \\
        &- n \tau \lr{k_{oxi} + k_{des}} \con{NH_3}^{ads} (k)\\
        &- n \tau k_{scr} \con{NO_x}^{in}(k) \con{NH_3}^{ads}(k)
\end{align*}

The above equation is nothing but multiplying the rate equation with the
sampling interval to get the measurement update.

% ==============================================================================

\subsubsection{Causal first-order hold approximation for $\sigma(k)$}

The term $\sigma$ is the average surface concentration of the adsorbed ammonia
within the sample time. It can be approximated as the average of the surface of
the previous sample duration. This is a causal approximation of the average
surface concentration:

\begin{align*}
    \sigma(k) &= \frac{1}{n} \sum_{i=0}^{n-1} \con{NH_3}^{ads} (k + i \tau)
                \approx \frac{1}{n+1} \sum_{i=0}^{n-1} \con{NH_3}^{ads} (k - i \tau)\\
              &\approx \frac{1}{2} \lr{\con{NH_3}^{ads} (k) + \con{NH_3}^{ads} (k - n \tau)}\\
\end{align*}
