\newpage
\section{$NO_x$ Process Dynamics}

The $NO_x$ process dynamics involve only the selective catalytic reduction
reaction. The gaseous $NO_x$ that enters the catalyst chamber reacts with the
adsorbed ammonia on the surface and forms $N_2$ and $H_2O$ (Eiley-Rideal
Mechanism). The process frees up the adsorption sites for the next cycle of
gasueous ammonia. The reaction reate is proportional to the volumetric concentration of the gaseous $NO_x$ and the surface concentration of the adsorbed ammonia.

\begin{align}
    4 NH_3 ^{ads} + 4 NO + O_2 &\xrightarrow[]{k_{scr}} 4 N_2 + 6 H_2O
\end{align}


Thus, the rate of $NO_x$ reduction on the catalyst surface can be modeled as:

\begin{align}
    \frac{d \con{NO_x}^{scr}}{dt} &= - k_{s2v} r_{scr} \\
    r_{scr} &= k_{scr} \con{NH_3}^{ads} \con{NO_x}^{in}\\
    \dot{\con{NO_x}}^{scr} &= -k_{s2v} k_{scr} \con{NH_3}^{ads} \con{NO_x}^{in}
\end{align}

% ==============================================================================

\subsection{Molar conservation at the scale of residence time}

Similar to the ammonia adsorption/desorption process, at the end of every
residence time, a fresh parcel of gaseous reactants enters the catalyst
chamber. Thus, the molar conservation for the $NO_x$ reduction process can be
will correlate the inlet and outlet molar concentrations of the $NO_x$ at the
beginning and the end of residence time. Thus,

\begin{align*}
    \mol{NO_x}^{out} (k + \tau) &= \mol{NO_x}^{in} (k) + V \int_{0}^{\tau}  \dot{\con{NO_x}}^{scr} (k) dt\\
    \mol{NO_x}^{out} (k + 2\tau) &= \mol{NO_x}^{in} (k + \tau) + V \int_{0}^{\tau} \dot{\con{NO_x}}^{scr} (k+\tau)  dt\\
    \vdots &\\
    \mol{NO_x}^{out} (k + n\tau) &= \mol{NO_x}^{in} (k + (n-1)\tau) + V \int_{0}^{\tau} \dot{\con{NO_x}}^{scr} (k+(n-1)\tau) dt
\end{align*}

The above equations show that the measurement of $NO_x$ concentration at the
outlet depends only on the measurement of the $NO_x$ concentration at the inlet
on residence time before. Thus, there is no integrating effect of the $NO_x$
within the sample time. Writing interms of volumetric concentrations, we have:

\begin{align*}
    \con{NO_x}^{out} (k + n\tau) &= \con{NO_x}^{in} (k + (n-1)\tau) + \int_{0}^{\tau}  \dot{\con{NO_x}}^{scr} (k + (n-1)\tau) dt
\end{align*}

Introducing the following two approximation:
\begin{enumerate}
    \item Zero-order-hold for the inlet concentration of $NO_x$:
        \begin{align*}
            \con{NO_x}^{in} (k + i \tau) &\approx \con{NO_x}^{in} (k) \qquad \forall i < n
        \end{align*}
    \item Using average surface concentration at the sample for the surface concentration of the adsorbed ammonia:
        \begin{align*}
            \con{NH_3}^{ads} (k + i \tau) &\approx \sigma(k) \qquad \forall i < n
        \end{align*}
\end{enumerate}

Thus,
\begin{align*}
    \dot{\con{NO_x}}^{scr} (k + i \tau) =  \dot{\con{NO_x}}^{scr} (k) = -k_{s2v} k_{scr} \sigma(k) \con{NO_x}^{in} (k) \qquad \forall i < n
\end{align*}


Thus, we have the following expression for the $NO_x$ process dynamics:

\begin{align*}
    \con{NO_x}^{out} (k + n\tau) &= \con{NO_x}^{in} (k) - k_{s2v} k_{scr} \sigma(k) \con{NO_x}^{in} (k) \tau\\
    \implies \con{NO_x}^{out} (k + 1) &= \con{NO_x}^{in} (k) \lr{1 - k_{s2v} k_{scr} \sigma(k) \tau}
\end{align*}


% ==============================================================================

\subsection{Preliminary aging signature}
The $NO_x$ process dynamics can be rewritten as:
\begin{align*}
    \frac{\con{NO_x}^{in}(k) - \con{NO_x}^{out}(k + 1)}{\con{NO_x}^{in}(k)} &=
    k_{s2v} k_{scr} \sigma(k) \tau\\
    &= \frac{A_{scr}}{V} \times k_{scr} \times \sigma(k) \lr{\frac{V \rho}{F}}\\
    %===
    \implies F \lr{\frac{\con{NO_x}^{in}(k) - \con{NO_x}^{out}(k + 1)}{\con{NO_x}^{in}(k)}} &= \lr{\rho A_{scr} k_{scr}} \sigma(k) \\
    %===
\end{align*}

Using the linear assumption for temperature dependence of the rate constant, and
ideal gas-law for density, let:
\begin{align*}
    k_{scr} &= m_{scr} T + c_{scr}\\
    \rho &= \frac{\mu}{T}
\end{align*}
T is in Kelvin.
\begin{align*}
    \implies \underbrace{\lr{\mu A_{scr} m_{scr}}\sigma}_{\alpha_1(k)}  T+ \underbrace{\lr{\mu A_{scr} c_{scr}} \sigma}_{\alpha_0(k)} &= \underbrace{T F \lr{\frac{\con{NO_x}^{in}(k) - \con{NO_x}^{out}(k + 1)}{\con{NO_x}^{in}(k)}}}_{\beta(k)}
\end{align*}

Thus, $\alpha_1(k)$ and $\alpha_0(k)$ are monotonic first order
polynomials in $\sigma$. Assuming, the average surface concentration of ammonia
on the catalyst is higher for degreened catalyst than for aged catalyst, we can
use $\alpha_1, \alpha_0$ to define a preliminary aging signature for the
catalyst. This can be used to monitor the aging of the catalyst in real-time.


The variables, $\alpha_1(k)$ and $\alpha_0(k)$ change with time. The only way to
estimate these variables is to estimate a moving-averaged version of it.


% ==============================================================================
We have the above equation for $n$ (Averaging window) samples:

\begin{align*}
    \alpha_1(k) T(k) + \alpha_0(k) &= \beta(k)\\
    \alpha_1(k-1)T(k-1) + \alpha_0(k-1) &= \beta(k-1)\\
    \vdots &\\
    \alpha_1(k-(n-1))T(k-(n-1)) + \alpha_0(k-(n-1)) &= \beta(k-(n-1))\\
\end{align*}

Let,
\begin{align*}
    \bar \alpha_1(k) &= \frac{1}{n} \sum_{i=0}^{n-1} \alpha_1(k-i)\\
    \bar \alpha_0(k) &= \frac{1}{n} \sum_{i=0}^{n-1} \alpha_0(k-i)\\
\end{align*}

Thus, we have the following equation:
\begin{align*}
    \bm{\bar \alpha_1(k)\\
        \bar \alpha_0(k)} &=
        \bm{T(k) & 1\\
            T(k-1) & 1\\
            \vdots & \vdots\\
            T(k-(n-1)) & 1}^{-1}
        \bm{\beta(k)\\
            \beta(k-1)\\
            \vdots\\
            \beta(k-(n-1))}
\end{align*}

