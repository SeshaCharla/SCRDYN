\newpage
\section{Urea Dosing Process Dynamics}
The urea dosing dynamics involve the following reactions:
\begin{align*}
    NH_2 - CO - NH_2 (liquid) &\longrightarrow NH_2 - CO - NH_2^* + x H_2 O
                & &[\text{AdBlue evaporation}] \\
    NH_2 - CO - NH_2^*  &\longrightarrow  HNCO + NH_3
                & &[\text{Urea decomposition}] \\
    HNCO + H_2O &\longrightarrow NH_3 + CO_2
                & &[\text{Isocynic acid hydrolysis}] \\
\end{align*}

The above reactions can be aggregated into:
\begin{align*}
    NH_2 - CO - NH_2 (liquid) &\longrightarrow 2 NH_3 + CO_2 + x H_2 O
\end{align*}

The rate of ammonia production is twice the rate of decomposition of the urea in the solution (AdBlue). Thus,
\begin{align*}
    \frac{d \con{NH_3}^{in}}{dt} &= 2 r_{u}\\
    r_{u} &= k_{u} \con{NH_2 - CO - NH_2} \qquad (constant)
\end{align*}

As the concentration of urea in the solution is constant, the above rate becomes a constant. Thus, the total moles of
ammonia produced at the inlet within a residence time would be:
\begin{align*}
    \mol{NH_3}^{in} (k) &= \tau \times \underbrace{2 r_{u}}_{\text{Rate of decomposition}} \times \underbrace{\tau u_{inj} (k)}_{\text{Volume injected}}
\end{align*}
Where $u_{inj}$ urea-solution injection rate in $ml/s$.

Thus, we have the inlet volumetric concentration of ammonia as:
\begin{align*}
    \con{NH_3}^{in} (k) &= \frac{\mol{NH_3}^{in} (k)}{V}
                          = \frac{2 r_u}{V} \times \tau^2 \times u_{inj} (k)
\end{align*}

Substituting, $\tau = \frac{V \rho_0}{F}$, we get:
\begin{align}
    \con{NH_3}^{in} (k) &= 2r_u V \rho^2_0 \times \frac{u_{inj}(k)}{F^2}
\end{align}


The above reciprocal relationship breaks down at zero flow rate which makes sense physically. Within the operating conditions we have:

% \begin{align*}
%     \delta  \con{NH_3}^{in} (k) &= -4r_u V \times \frac{\bar u_{inj}}{\bar{f}_v^3} \delta f_v
%                                    + 2r_u V \times \frac{\delta u_{inj}}{\bar f_v^2}\\
% \end{align*}
%
% The linearized approximation model would be:
%
% \begin{align*}
%     \con{NH_3}^{in}(k) &= \con{NH_3}^{in}_0 + \delta \con{NH_3}^{in} (k)\\
%                        &= 2r_u V \times \frac{\bar u_{inj}}{ \bar f_v^2}
%                          + 2r_u V \times \frac{\delta u_{inj}}{\bar f_v^2}
%                          -4r_u V \times \frac{\bar u_{inj}}{\bar{f}_v^3} \delta f_v\\
%                        &= 2r_u V \times \frac{u_{inj}(k)}{ \bar f_v^2}
%                          -4r_u V \times \frac{\bar u_{inj}}{\bar{f}_v^3} \delta f_v(k)
%                          \qquad \lrb{\because \bar u_{inj} + \delta u_{inj} = u_{inj}}
% \end{align*}
% Introducing the $\delta f_v$ as function of $F$ and $T$ from equation \ref{eqn::fv_approx},
% \begin{align*}
%     \con{NH_3}^{in}(k) &= \lr{\frac{2r_u V}{\bar f_v^2}} u_{inj}(k)
%                           -\lr{\frac{ 4r_u V \bar u_{inj}}{\bar{f}_v^3}}\lr{\frac{1}{\mu} \lr{F_0 \delta T + T_0 \delta F} } \\
%                        &= \lr{\frac{2r_u V}{\bar f_v^2}} u_{inj}(k)
%                           -\lr{\frac{ 4r_u V \bar u_{inj} F_0 }{\mu \bar{f}_v^3}} \delta T
%                           -\lr{\frac{ 4r_u V \bar u_{inj} T_0 }{\mu \bar{f}_v^3}} \delta F
% \end{align*}
% Dropping the $\delta$ for notational convenience,
\begin{align}
    \con{NH_3}^{in}(k) &= \nu_u \frac{u_{inj}}{F^2}    \label{eqn::urea_inj}
\end{align}
where,
\begin{align*}
    \nu_u &= 2r_u V \rho^2_0\\
    T_{min} &\leq T \leq T_{max}\\
    F_{min} &\leq F \leq F_{max}
\end{align*}
