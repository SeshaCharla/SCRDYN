\newpage
\section{Urea Dosing Process Dynamics}
The urea dosing dynamics involve the following reactions:
\begin{align*}
    NH_2 - CO - NH_2 (liquid) &\longrightarrow NH_2 - CO - NH_2^* + x H_2 O
                & &[\text{AdBlue evaporation}] \\
    NH_2 - CO - NH_2^*  &\longrightarrow  HNCO + NH_3
                & &[\text{Urea decomposition}] \\
    HNCO + H_2O &\longrightarrow NH_3 + CO_2
                & &[\text{Isocynic acid hydrolysis}] \\
\end{align*}

The above reactions can be aggregated into:
\begin{align*}
    NH_2 - CO - NH_2 (liquid) &\longrightarrow 2 NH_3 + CO_2 + x H_2 O
\end{align*}

The rate of ammonia production is twice the rate of decomposition of the urea in the solution (AdBlue). Thus,
\begin{align*}
    \frac{d \con{NH_3}^{in}}{dt} &= 2 r_{u}\\
    r_{u} &= k_{u} \con{NH_2 - CO - NH_2} \qquad (constant)
\end{align*}

As the concentration of urea in the solution is constant, the above rate becomes a constant. Thus, the total moles of
ammonia produced at the inlet within a residence time would be:
\begin{align*}
    \mol{NH_3}^{in} (k) &= \tau \times \underbrace{2 r_{u}}_{\text{Rate of decomposition}} \times \underbrace{\tau u_{inj} (k)}_{\text{Volume injected}}
\end{align*}
Where $u_{inj}$ urea-solution injection rate in $ml/s$.

Thus, we have the inlet volumetric concentration of ammonia as:
\begin{align*}
    \con{NH_3}^{in} (k) &= \frac{\mol{NH_3}^{in} (k)}{V}
                          = \frac{2 r_u}{V} \times \tau^2 \times u_{inj} (k)
\end{align*}

Substituting, $\tau = \frac{V}{f_v}$, we get:
\begin{align}
    \con{NH_3}^{in} (k) &= 2r_u V \times \frac{u_{inj}(k)}{f_v^2}
\end{align}
