\subsection{Density of the exhaust gas}

The density of the exhaust flow is assumed to be the density of air at that
temperature and ambient atmospheric pressure. Using the ideal gas law:
\begin{align}
    \rho &= \frac{PM}{R T} = \frac{\mu}{T}
\end{align}
\begin{align*}
    \text{where, } &\\
    P &= \text{Pressure of the exhaust gas (ambient pressure)} = 101.325 \: kPa\\
    M &= \text{Molecular weight of the exhaust gas} = 28.9652 \: g/mol\\
    T &= \text{Temperature of the exhaust gas in Kelvin}\\
    R &= \text{Universal gas constant} = 8.314 \: J/(mol.K)\\
\end{align*}

\itbf{Note}: $P$ and $M$ are replaced with $(P_1 M_1 + P_2 M_2)$ when humidity of the exhaust gas is considered.

\subsubsection{$\%$ Change in density for the temperature range of operation}
We have,
\begin{align*}
    \frac{\delta \rho}{\rho} &= \frac{\delta T}{T}
\end{align*}
In general the operating temperature is very his ($250 \,^0 C \approx 500 K$) and most of the data lies in $\pm 50 \, ^0 C$ region. Thus, the maximum change in density is less than $10\%$ whose effect on flow rate is far smaller as compared to the change in mass flow rate itself. Thus, density can be assumed to be a constant for this process.

\begin{align}
    \rho &= \rho_0
\end{align}
