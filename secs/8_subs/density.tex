\subsection{Density of the exhaust gas}

The density of the exhaust flow is assumed to be the density of air at that
temperature and ambient atmospheric pressure. Using the ideal gas law:
\begin{align}
    \rho &= \frac{PM}{R T} = \frac{\mu}{T}
\end{align}
\begin{align*}
    \text{where, } &\\
    P &= \text{Pressure of the exhaust gas (ambient pressure)} = 101.325 \: kPa\\
    M &= \text{Molecular weight of the exhaust gas} = 28.9652 \: g/mol\\
    T &= \text{Temperature of the exhaust gas in Kelvin}\\
    R &= \text{Universal gas constant} = 8.314 \: J/(mol.K)\\
\end{align*}

\itbf{Note}: $P$ and $M$ are replaced with $(P_1 M_1 + P_2 M_2)$ when humidity of the exhaust gas is considered.

\subsubsection{$\%$ Change in density for the temperature range of operation}
We have,
\begin{align*}
    \frac{\delta \rho}{\rho} &= -\frac{\delta T}{T}
\end{align*}
In general the operating temperature is very high ($250 \,^0 C \approx 500 K$), and most of the data lies in $\pm 100 \, ^0 C$ region. Thus, the maximum change in density is less than $10\%$, whose effect on flow rate is far smaller compared to the change in mass flow rate itself. Thus, density can be assumed to be a constant for this process.
\begin{align}
    \rho &= \rho_0
\end{align}

%%==================

\subsubsection{Sensitivity of density to change in $NO_x$ concentrations}
Let, $P_{NO_x}$ be the partial pressure of $NO_x$ and $P_0$ be the total pressure. Assuming the rest of the gas constitutions are similar to that of air. We have the density as:
\begin{align*}
    \rho &= \frac{1}{RT} \lr{ P_{NO_x} M_{NO_x} + (P_0 - P_{NO_x}) M_{air}} = \frac{1}{RT} \lr{ P_{NO_x} \lr{M_{NO_x} - M_{air}} + P_0 M_{air}}\\
    \implies \frac{\partial \rho}{\partial P_{NO_x}} &= \frac{1}{RT} \lr{\lr{M_{NO_x} - M_{air}}}\\
    \implies \frac{\delta \rho}{\rho} &= \frac{\lr{\lr{M_{NO_x} - M_{air}} \delta P_{NO_x}}}{\lr{ P_{NO_x} \lr{M_{NO_x} - M_{air}} + P_0 M_{air}}} \qquad \text{at constant temperature}
\end{align*}
Also,
\begin{align*}
    P_{NO_x} &= P_0 \times \frac{n_{NO_x}}{n_{NO_x} + n_{air}} = P_0 \times \frac{\con{NO_x}}{\frac{n_{NO_x} + n_{air}}{V}} = P_0 \times \frac{\con{NO_x}}{N_a (=1)} \qquad \lrb{\because 1 \, mole = n_{tot}/V}
\end{align*}
Thus,
\begin{align*}
    \frac{\delta \rho}{\rho} &= \frac{\lr{\lr{M_{NO_x} - M_{air}} \delta \con{NO_x}}}{\lr{ \con{NO_x}
 \lr{M_{NO_x} - M_{air}} + M_{air}}}
    \qquad \text{at constant temperature}
\end{align*}
We have,
\begin{align*}
    &M_{NO_x} \approx 46 \, g/mol, \qquad M_{air} \approx 29 \, g/mol
\end{align*}
\begin{align*}
    \implies \frac{\delta \rho}{\rho} &= \frac{\lr{ 17 \delta \con{NO_x}}}{17 \con{NO_x} + 29}
    \qquad \text{at constant temperature}
\end{align*}

The above equation demonstrates that the relative change in density due to relative change in $NO_x$ concentration is insignificant. As the molarity of $NO_x$ itself is small.
