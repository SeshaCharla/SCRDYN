\subsection{Parts-per-million to mol/m$^3$}

The ppm measurements of concentration is, by convention, assumed to be the
weight of the solute in grams per million grams of solution (water). This can be
converted to the concentration of the solute in mol/ml using the molecular
weight of the solute and the density of the solution $(1 g/ml)$. We have,

\begin{align*}
    1 \, ppm &= \frac{1 \, g}{10^6 \, ml} = 1 \, mg/L = \frac{10^{-3} \, g}{10^{-3} \, m^3} = 1 \, g/m^3\\
\end{align*}

Let M be the molecular weight of the solute in g/mol. Then,
\begin{align}
    1 \, ppm &= \frac{1}{M} \, mol/m^3
\end{align}


\begin{align*}
    1 \, ppm^{\lr{mass}} &= \frac{1 \, g\text{ of gas}}{10^6 \, g \text{ of air }}
\end{align*}

\begin{align*}
    1 \, ppm^{\lr{mol}} &= \frac{1 \, mol \text{ of gas}}{10^6 \, mol \text{ of air }}
\end{align*}


\begin{align*}
    1 \, ppm ^{\lr{vol}} &= \frac{1 \, ml \text{ of gas}}{10^6 \, ml \text{ of air}}
\end{align*}
