\subsection{Mass flow rate to volumetric flow rate}
The temperature dependent density of the exhaust gas is used to convert the mass
flow rate of the exhaust gas to the volumetric flow rate of the exhaust gas.
\begin{align}
    f_v &= \frac{F}{\rho} = \frac{FT}{\mu}\\
    \implies \delta f_v &= \frac{1}{\mu} \lr{F_0 \delta T + T_0 \delta F}   \label{eqn::fv_approx}
\end{align}


\subsubsection{Linear model for residence time}
We have the residence time of the exhaust gas in the SCR-ASC system as:
\begin{align*}
    \tau &= \frac{V}{f_v} = \frac{V \mu}{FT} \\
    \implies
    \delta \tau &= - \frac{V}{\bar{f_v} ^2} \delta f_v
\end{align*}
Thus we have the linear approximation of the residence time:
\begin{align*}
    \tau &= \tau_0 + \delta \tau
          = \tau_0 - \frac{V}{\bar{f_v} ^2} \delta f_v\\
         &= \tau_0 - \frac{V}{\bar{f_v} ^2}  \frac{1}{\mu} \lr{F_0 \delta T + T_0 \delta F}
          = \tau_0 - \tau_T \delta T - \tau_F \delta F
\end{align*}
dropping $\delta$ for notational convenience, we have the linear model for residence time:
\begin{align}
    \tau &= \tau_0 - \tau_T T - \tau_F F        \label{eqn::res_time}
\end{align}
Where,
\begin{align*}
    \tau_T &= \frac{V}{\bar{f_v} ^2}  \frac{T_0}{\mu}\\
    \tau_F &= \frac{V}{\bar{f_v} ^2}  \frac{F_0}{\mu}\\
    F_{min} &\leq F \leq F_{max}\\
    T_{min} &\leq T \leq T_{max}
\end{align*}
