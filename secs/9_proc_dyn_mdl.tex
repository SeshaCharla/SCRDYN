\newpage
\section{SCR Process Dynamics with Urea Dosing \label{sec::proc_dyn}}
From previous derivations, we have the complete process dynamics of SCR-ASC dynamics with urea dosing as:

\begin{enumerate}
    \item Urea dosing dynamics:
    \begin{align*}
    \con{NH_3}^{in} (k) &= 2r_u V \times \frac{u_{inj}(k)}{f_v^2}
    \end{align*}

    \item $NO_x$ reduction dynamics:
    \begin{align*}
    \con{NO_x}^{out} (k + 1) &= \con{NO_x}^{in} (k) \lr{1 - k_{s2v} k_{scr} \sigma(k) \tau}
    \end{align*}

    \item Gaseous ammonia dynamics:
    \begin{align*}
    \con{NH_3}^{out} (k + 1) &= \con{NH_3}^{in}(k) \lr{1 - \tau k_{s2v}k_{ads} \lr{\Gamma - \sigma(k)}} + \tau k_{s2v} k_{des} \sigma(k)
    \end{align*}

    \item Ammonia adsorption/desorption dynamics:
 \begin{align*}
        \sigma(k + 1) &= \sigma(k)
        + n\tau k_{ads} \con{NH_3}^{in} \lr{\Gamma - \sigma(k)}
        - n \tau \lr{k_{oxi} + k_{des}} \sigma(k)
        - n \tau k_{scr} \con{NO_x}^{in}(k) \sigma(k)
    \end{align*}
\end{enumerate}

The above four equations are the complete process dynamics of SCR-ASC dynamics with urea dosing. The above equations are
nonlinear and coupled with implicit dependence on temperature and flow rate. In this section the above equations are
rewritten in a parametric form with standard state-space representation. For convenience and clarity '(k)' is dropped
from the variables.

\subsection{Linear Parameter Models}
The following states and inputs are defined for the system:

\begin{align*}
    x_1 &= \con{NO_x}^{out} & u_1 &= \con{NO_x}^{in} \\
    x_2 &= \con{NH_3}^{out} & u_2 &= u_{inj} \\
    x_3 &= \sigma & u_3 &= T\\
        &         & u_4 &= F
\end{align*}

Thus, linear-parameter representation of each of the processes are as follows:

\subsection{Urea Dosing Dynamics}
We have,
\begin{align}
    \con{NH_3}^{in}(k) &= \nu_u \frac{u_{inj}}{F^2}   \label{eqn::urea_parm}
\end{align}
where,
\begin{align*}
    \nu_u &= 2r_u V \rho^2_0\\
    T_{min} &\leq T \leq T_{max}\\
    F_{min} &\leq F \leq F_{max}
\end{align*}

\begin{align*}
        \pmb \phi_{ur} &= \bm{\frac{u_{inj}}{F^2}}\\
        \pmb \theta_{ur} &= \bm{\nu_u = 2r_u V \rho^2_0}
\end{align*}

\subsection{NOx Reduction Dynamics}
We have:
\begin{align*}
    \con{NO_x}^{out} (k + 1) &= \con{NO_x}^{in} (k) \lr{1 - k_{s2v} k_{scr} \sigma(k) \tau}\\
    \implies x_1 (k+1) &= u_1 \lr{1 - k_{s2v} k_{scr} x_3 \tau}\\
                       &= u_1 \lrf{1 + k_{s2v} x_3 \pmb \phi_\tau^T \pmb \theta_{scr}} \qquad [\because \ref{eqn::tau_ki}]
\end{align*}
Writing in linear regression form, let
\begin{align*}
    \pmb \phi_1^T &= u_1 \pmb \phi_\tau^T = u_1 \bm{T^2 &  T F & - T &  F & -1} \\
    \pmb \theta_1^T &= k_{s2v} \pmb \theta_{scr}^T = k_{s2v} \bm{m_{scr} \tau_T &
                                                               m_{scr} \tau_F &
                                                               \lr{ m_{scr} \tau_0  -  \tau_T c_{scr} } &
                                                               c_{scr} \tau_F &
                                                               c_{scr} \tau_0}
\end{align*}
\begin{align}
    x_1(k+1) &= u_1(k) + \pmb \phi_1^T (k) \pmb \theta_1 x_3  \label{eqn::nox_regression}
\end{align}



% ======================================================================================================================












%% Old Stuff
%                     &= u_1 \lr{1 - k_{s2v}\lr{q_{scr} T^2 + m_{scr}T + c_{scr}} x_3 \lr{\frac{V}{f_v}}}\\
%                     %===
%                     &= u_1 \lr{1 - A_{scr}\lr{ q_{scr} T^2 + m_{scr}T + c_{scr}} \lr{\frac{x_3}{f_v}}}\\
%                     &= u_1 \lr{1 - A_{scr}\mu \lr{ q_{scr} T^2 + m_{scr}T + c_{scr}} \lr{\frac{x_3}{F T}}}\\
%                     &= u_1 - \underbrace{\lr{A_{scr} \mu q_{scr}}}_{\theta_{11}} \lr{ \frac{x_3 T u_1}{F}}
%                                  - \underbrace{A_{scr} \mu m_{scr}}_{\theta_{12}} \lr{ \frac{x_3 u_1}{F}}
%                                  - \underbrace{A_{scr} \mu c_{scr}}_{\theta_{13}} \lr{ \frac{x_3 u_1}{FT}}
% \end{align*}
% \begin{align}
%     x_1(k+1)  &= u_1 - \theta_{11} \lr{\frac{u_1 u_3 x_3}{u_4}}
%                                      - \theta_{12} \lr{\frac{u_1 x_3}{u_4}}
%                                      - \theta_{13} \lr{\frac{u_1 x_3}{u_3 u_4}}\\
%     %===
%     \implies x_1(k + 1) &= u_1 - \pmb \phi_1^T \lr{x_3 \pmb \theta_1} \label{eq::nox_proc}
% \end{align}
% where:
% \begin{align*}
%     \pmb \phi_1 &= \bm{\frac{u_1 u_3}{u_4} & \frac{u_1}{u_4} & \frac{u_1}{u_3 u_4}}^T\\
%     \pmb \theta_1 &= \bm{\theta_{11} & \theta_{12} & \theta_{13}}^T
% \end{align*}
% %===
% Let,
% \begin{align}
%     \pmb \phi_0 &= \bm{\frac{u_3}{u_4} & \frac{1}{u_4} & \frac{1}{u_3 u_4}}^T
% \end{align}
% \begin{align*}
%     \implies \pmb \phi_1 &= u_1 \pmb \phi_0
% \end{align*}
%

\subsection{NH$_3$ Gas Dynamics}

We have:
\begin{align*}
    \con{NH_3}^{out} (k + 1) &= \con{NH_3}^{in}(k) \lr{1 - \tau k_{s2v}k_{ads} \lr{\Gamma - \sigma(k)}} + \tau k_{s2v} k_{des} \sigma(k)\\
    %===
    \implies x_2 (k+1) &= \con{NH_3}^{in} \lrf{1 - \tau k_{s2v}k_{ads} \lr{\Gamma - x_3}} + \tau k_{s2v} k_des x_3\\
                       &= \con{NH_3}^{in} - \con{NH_3}^{in} \tau k_{s2v}k_{ads} \lr{\Gamma - x_3} + \tau k_{s2v} k_{des} x_3\\
                       &= \nu_u \frac{u_{inj}}{F}
                            - k_{s2v} \pmb \phi_{\tau, ur}^T \lr{\tau_0 \nu_u \pmb \theta_{ads}} \lr{\Gamma - x_3}
                            + k_{s2v} \pmb \phi_{\tau}^T \lr{\tau_0 \pmb \theta_{des}} x_3
                        \qquad \lrb{\because \ref{eqn::urea_parm}, \ref{eqn::k_tau_urea}, \ref{eqn::tau_ki}}\\
                        %===
        &= \bm{ \frac{u_{inj}}{F} & -\pmb \phi_{\tau, ur}^T }
           \bm{ \nu_u  \\  \Gamma k_{s2v} \tau_0 \nu_u \pmb \theta_{ads}  }
           + x_3
           \bm{\pmb \phi_{\tau, ur}^T & + \pmb \phi_{\tau}^T }
           \bm{k_{s2v} \tau_0 \nu_u \pmb \theta_{ads} \\  k_{s2v} \tau_0 \pmb \theta_{des}}
\end{align*}
%==
Writing in linear regression form, let,
\begin{align*}
    \pmb \phi_{20}^T &= \bm{\frac{u_{inj}}{F} & -\pmb \phi_{\tau, ur}^T } \qquad
    \pmb \phi_2^T    = \bm{\pmb \phi_{\tau, ur}^T & + \pmb \phi_{\tau}^T } \\
    \pmb \theta_{20} &= \bm{\nu_u \\  \Gamma k_{s2v} \tau_0 \nu_u \pmb \theta_{ads}} \qquad
    \pmb \theta_2    = \bm{k_{s2v} \tau_0 \nu_u \pmb \theta_{ads} \\  k_{s2v} \tau_0 \pmb \theta_{des}}
\end{align*}
%==
Thus,
\begin{align}
    x_2 (k+1) &= \pmb \phi_{20}^T \pmb \theta_{20} + x_3 \pmb \phi_2^T \pmb \theta_2
    \label{eqn::nh3_gas_regression}
\end{align}
















%%%%%%%%%%%%%%%%%%%%%%%%%%
%% Slightly Older Stuff %%
%%%%%%%%%%%%%%%%%%%%%%%%%%

%                        &= \bm{\pmb \phi_{ur}^T & -\pmb \phi_{\tau, ur}^T  & -\pmb \phi_\tau^T}
%                           \bm{I_3    & \pmb 0                   & \pmb 0 \\
%                               \pmb 0 & \lr{\Gamma - x_3} I_{11} & \pmb 0 \\
%                               \pmb 0 & \pmb 0                   & x_3 I_5}
%                           \bm{\pmb \theta_{ur} \\
%                               k_{s2v} \pmb \theta_{ads, ur} \\
%                               k_{s2v} \pmb \theta_{des} }
% \end{align*}
% Where, $I_n$ is identity matrix of size n.
%
% Writing in linear regression form, let,
% \begin{align*}
%     \pmb \phi_2^T &= \bm{\pmb \phi_{ur}^T & -\pmb \phi_{\tau, ur}^T  & -\pmb \phi_\tau^T}\\
%     \pmb \theta_2 &= \bm{\pmb \theta_{ur} \\
%                         k_{s2v} \pmb \theta_{ads, ur} \\
%                         k_{s2v} \pmb \theta_{des} }       \\
%     \pmb \Psi_2   &= \bm{I_3    & \pmb 0                   & \pmb 0 \\
%                       \pmb 0 & \lr{\Gamma - x_3} I_{11} & \pmb 0 \\
%                       \pmb 0 & \pmb 0                   & x_3 I_5}
% \end{align*}
% \begin{align}
%     x_2(k+1) &= \pmb \phi_2 ^T \Psi_2 \pmb \theta_2   \label{eqn::nh3_gas_regression}
% \end{align}
%













%%%%%%%%%%%%%%
%% Old Stuff %
%%%%%%%%%%%%%%
%     %===
%     \implies x_2(k+1) &= b_u \lr{\frac{u_2}{(u_3u_4)^2}} \lr{ 1 - \lr{\frac{A_{scr} \mu}{FT}}  \lr{q_{ads}T^2 + m_{ads} T + c_{ads}} \lr{\Gamma - x_3}}\\
%                                  &\qquad + \lr{\frac{A_{scr} \mu}{FT}} \lr{q_{des} T^2 + m_{des} T + c_{des}} x_3\\
%     %===
%     &= b_u \lr{\frac{u_2}{(u_3u_4)^2}}
%                     \lr{ 1 - \lrf{\lr{A_{scr} \mu q_{ads}} \lr{\frac{T}{F}} + \lr{A_{scr} \mu m_{ads}} \lr{\frac{1}{F}} + \lr{A_{scr} \mu c_{ads}}  \lr{\frac{1}{FT}}}
%                     \lr{\Gamma - x_3}}\\
%                                  &\qquad + \lr{A_{scr} \mu q_{des}} \lr{\frac{T x_3}{F}} + \lr{A_{scr} \mu m_{des}} \lr{\frac{x_3}{F}} + \lr{A_{scr} \mu c_{des}} \lr{\frac{x_3}{FT}}\\
%     %===
%     &= \underbrace{b_u}_{\theta{20}} \lr{\frac{u_2}{(u_3u_4)^2}}
%     + \underbrace{\lr{A_{scr} \mu q_{des}}}_{\theta_{21}}  \lr{\frac{u_3 x_3}{u_4}}
%     + \underbrace{\lr{A_{scr} \mu m_{des}}}_{\theta{22}}   \lr{\frac{x_3}{u_4}}
%     + \underbrace{\lr{A_{scr} \mu c_{des}}}_{\theta{23}}  \lr{\frac{x_3}{u_3 u_4}}\\
%     %===
%     &\qquad - \lr{\Gamma - x_3}  \lrf{\underbrace{\lr{A_{scr} \mu q_{ads}b_u}}_{\theta_{24}} \lr{\frac{u_2}{u_3 u_4^3}}
%                                                          + \underbrace{\lr{A_{scr} \mu m_{ads}b_u}}_{\theta_{25}} \lr{\frac{u_2}{u_3^2 u_4^3}}
%                                                          + \underbrace{\lr{A_{scr} \mu c_{ads} b_u}}_{\theta_{26}}  \lr{\frac{u_2}{\lr{u_3 u_4}^3}}}
% \end{align*}
% \begin{multline}
%     \therefore x_2(k+1)
%     %===
%     = \theta_{20} \lr{\frac{u_2}{(u_3u_4)^2}}
%     + \theta_{21}  \lr{\frac{u_3 x_3}{u_4}}
%     + \theta_{22}   \lr{\frac{x_3}{u_4}}
%     + \theta_{23}  \lr{\frac{x_3}{u_3 u_4}}\\
%     %===
%     + \theta_{24} \lr{\frac{u_2}{u_3 u_4^3}} x_3
%     + \theta_{25} \lr{\frac{u_2}{u_3^2 u_4^3}} x_3
%     + \theta_{26}  \lr{\frac{u_2}{\lr{u_3 u_4}^3}} x_3\\
%     %===
%     - \underbrace{\Gamma \theta_{24}}_{\theta_{28}} \lr{\frac{u_2}{u_3 u_4^3}}
%     - \underbrace{\Gamma \theta_{25}}_{\theta_{28}} \lr{\frac{u_2}{u_3^2 u_4^3}}
%     - \underbrace{\Gamma \theta_{26}}_{\theta_{29}}  \lr{\frac{u_2}{\lr{u_3 u_4}^3}}
%     %===
% \end{multline}
% We have the regression model:
% \begin{align}
%    x_3(k+1) &= \pmb \phi_2^T \pmb \theta_2 \label{eq::nh3_proc}\\
%    \pmb \theta_2 &= \bm{\theta_{20} & \theta_{21} & \hdots & \theta_{29}}^T \\
%    \pmb \phi_2 &= \bm{\frac{u_2}{(u_3u_4)^2} & \vline & x_3 \lr{\frac{u_2}{(u_3 u_4)^2}} \pmb \phi_0^T & \vline & - \lr{\frac{u_2}{(u_3 u_4)^2}} \pmb \phi_0^T}
% \end{align}
%

\subsection{$NH_3$ adsorption/desorption process dynamics}
We have,
\begin{align*}
        \sigma(k + 1) &= \sigma(k)
        + n\tau k_{ads} \con{NH_3}^{in} \lr{\Gamma - \sigma(k)}
        - n \tau \lr{k_{oxi} + k_{des}} \sigma(k)
        - n \tau k_{scr} \con{NO_x}^{in}(k) \sigma(k) \\
        %===
        \implies x_3(k+1) &= x_3 + n \pmb \phi_{\tau, ur} \pmb \theta_{ads, ur} \lr{\Gamma - x_3}
                                 - n \pmb \phi_{\tau} \pmb \theta_{od} x_3
                                 - n \pmb \phi_{\tau} \pmb \theta_{scr} u_1 x_3
                                 \qquad \lrb{\because \ref{eqn::k_sum}, \ref{eqn::tau_ki}, \ref{eqn::k_tau_urea}} \\
        %===
        &= x_3  \bm{1 &
                   - \pmb \phi_{\tau, ur} &
                   - \pmb \phi_\tau  &
                   - u_1 \pmb \phi_{\tau}}
        \bm{ 1\\
            n \pmb \theta_{ads, ur}    \\
            n \pmb \theta_{od}         \\
            n \pmb \theta_{scr}}
        + \pmb \phi_{\tau, ur} \lrf{n \Gamma \pmb \theta_{ads, ur} }
\end{align*}
Writing in linear regression form, let,
\begin{align*}
        \pmb \phi_3 &= \bm{1&
                           - \pmb \phi_{\tau, ur} &
                           - \pmb \phi_\tau  &
                           - u_1 \pmb \phi_{\tau}} \\
        \pmb \theta_3 &= \bm{1 \\
                             n \pmb \theta_{ads, ur}    \\
                             n \pmb \theta_{od}         \\
                             n \pmb \theta_{scr}} , \qquad
        \pmb \theta_\Gamma  = n \Gamma \pmb \theta_{ads, ur}
\end{align*}
Thus,
\begin{align}
        x_3(k+1) &= x_3 \pmb \phi_3^T \theta_3 + \pmb \phi_{\tau, ur} \pmb \theta_\Gamma
        \label{eqn::nh3_ads_regression}
\end{align}























%%%%%%%%%%%%%%%%%%%
%% Old Stuff %%%%%%
%%%%%%%%%%%%%%%%%%%

% \begin{align*}
%         x_3(k+1) &= x_3
%                 + n \lr{\frac{V \mu}{FT}} \lr{q_{ads} T^2 + m_{ads} T + c_{ads}} \lr{b_u \lr{\frac{u_2}{u_3^2 u_4^2}}} \lr{\Gamma - x_3} \\
%                 & \qquad - n \lr{\frac{V \mu}{FT}} \lr{q_{od} T^2 + m_{od} T + c_{od}} x_3 \\
%                 & \qquad - n \lr{\frac{V \mu}{FT}} \lr{q_{scr} T^2 + m_{scr} T + c_{scr}} u_1 x_3\\
%                 %===
%                 &= x_3 - \underbrace{\lr{n V \mu q_{od}}}_{\theta_{31}} \lr{\frac{x_3 u_3}{u_4}}
%                        - \underbrace{\lr{n V \mu m_{od}}}_{\theta_{32}} \lr{\frac{x_3}{u_4}}
%                        - \underbrace{\lr{n V \mu c_{od}}}_{\theta_{33}} \lr{\frac{x_3}{u_3 u_4}} \\
%                 & \qquad - \underbrace{\lr{n V \mu q_{scr}}}_{\theta_{34}} \lr{\frac{x_3 u_1 u_3}{u_4}}
%                          - \underbrace{\lr{n V \mu m_{scr}}}_{ \theta_{35}}\lr{\frac{x_3 u_1}{u_4}}
%                          - \underbrace{\lr{n V \mu c_{scr}}}_{\theta_{36}} \lr{\frac{x_3 u_1}{u_3 u_4}} \\
%                 & \qquad +\lrf{ \underbrace{\lr{n V \mu b_u q_{ads}} }_{\theta_{37}}\lr{\frac{u_2}{u_3 u_4^3}}
%                               + \underbrace{\lr{n V \mu b_u m_{ads}} }_{\theta_{38}}\lr{\frac{u_2}{u_3^2 u_4^3}}
%                               + \underbrace{\lr{n V \mu b_u c_{ads}} }_{\theta_{39}}\lr{\frac{u_2}{u_3^3 u_4^3}} } \lr{\Gamma - x_3}
% \end{align*}
% \begin{multline}
%         x_3(k+1) = x_3 - {\theta_{31}} \lr{\frac{x_3 u_3}{u_4}}
%                                      - {\theta_{32}} \lr{\frac{x_3}{u_4}}
%                                      - {\theta_{33}} \lr{\frac{x_3}{u_3 u_4}} \\
%                 - {\theta_{34}} \lr{\frac{x_3 u_1 u_3}{u_4}}
%                 -{\theta_{35}}\lr{\frac{x_3 u_1}{u_4}}
%                 - {\theta_{36}} \lr{\frac{x_3 u_1}{u_3 u_4}} \\
%                 - {\theta_{37}}\lr{\frac{u_2}{u_3 u_4^3}} x_3
%                 - {\theta_{38}}\lr{\frac{u_2}{u_3^2 u_4^3}} x_3
%                 - {\theta_{39}}\lr{\frac{u_2}{u_3^3 u_4^3}} x_3 \\
%                 + \underbrace{\Gamma \theta_{37}}_{\theta_{3a}} \lr{\frac{u_2}{u_3 u_4^3}}
%                 + \underbrace{\Gamma \theta_{38}}_{\theta_{3b}} \lr{\frac{u_2}{u_3^2 u_4^3}}
%                 + \underbrace{\Gamma \theta_{39}}_{\theta_{3c}} \lr{\frac{u_2}{u_3^3 u_4^3}}
% \end{multline}
% \begin{align}
%         \implies x_3(k+1) &= x_3 - \pmb \phi_3^T \pmb \theta_3 \label{eq::nh3_ads}
% \end{align}
% where:
% \begin{align}
%         \pmb \phi_3 &= \bm{
%                 \frac{x_3 u_3}{u_4}       & \frac{x_3}{u_4}             & \frac{x_3}{u_3 u_4}         &
%                 \frac{x_3 u_1 u_3}{u_4}   & \frac{x_3 u_1}{u_4}         & \frac{x_3 u_1}{u_3 u_4}     &
%                 \frac{x_3 u_2}{u_3 u_4^3} & \frac{x_3 u_2}{u_3^2 u_4^3} & \frac{x_3 u_2}{u_3^3 u_4^3} &
%                 -\frac{u_2}{u_3 u_4^3}    & -\frac{u_2}{u_3^2 u_4^3}    & -\frac{u_2}{u_3^3 u_4^3} }^T\\
%         \pmb \theta_3 &= \bm{\theta_{31} &
%                              \theta_{32} &
%                              \theta_{33} &
%                              \theta_{34} &
%                              \theta_{35} &
%                              \theta_{36} &
%                              \theta_{37} &
%                              \theta_{38} &
%                              \theta_{39} &
%                              \theta_{3a} &
%                              \theta_{3b} &
%                              \theta_{3c}}^T
% \end{align}
% Rewriting $\pmb \phi_3$ interms of $\pmb \phi_0$:
% \begin{align}
%         \pmb \phi_3 &= \bm{x_3 \pmb \phi_0^T & \vline & x_3 u_1 \pmb \phi_0 ^T & \vline & x_3 \lr{\frac{u_2}{u_3^2 u_4^2}} \pmb \phi_0 ^T & \vline & - \lr{\frac{u_2}{u_3^2 u_4^2}} \pmb \phi_0 ^T}^T
% \end{align}
%

% ==============================================================================
\subsection{SCR-ASC Process Dynamics in Parametric Model Identification}
\begin{align*}
    x_1(k+1) &= u_1 - \pmb \phi_1^T \lr{x_3 \pmb \theta_1}\\
    x_2(k+1) &= \pmb \phi_2^T \pmb \theta_2\\
    x_3(k+1) &= x_3 - \pmb \phi_3^T \pmb \theta_3\\
    \text{where:} \qquad &\\
    \pmb \phi_1 &= u_1 \pmb \phi_0\\
    \pmb \phi_2 &= \bm{\frac{u_2}{(u_3u_4)^2} & \vline & x_3 \lr{\frac{u_2}{(u_3 u_4)^2}} \pmb \phi_0^T & \vline & - \lr{\frac{u_2}{(u_3 u_4)^2}} \pmb \phi_0^T}^T\\
    \pmb \phi_3 &= \bm{x_3 \pmb \phi_0^T & \vline & x_3 u_1 \pmb \phi_0 ^T & \vline & x_3 \lr{\frac{u_2}{u_3^2 u_4^2}} \pmb \phi_0 ^T & \vline & - \lr{\frac{u_2}{u_3^2 u_4^2}} \pmb \phi_0 ^T}^T\\
    \pmb \phi_0 &= \bm{\frac{u_3}{u_4} & \frac{1}{u_4} & \frac{1}{u_3 u_4}}^T
\end{align*}


% ==============================================================================
\subsubsection{Windowed Least Squares For Parameter Estimation}

As gaseous $NH_3$ out measurements don't affect the $NO_x$ and $NH_3^{(ads)}$ process dynamics, only $NO_x$ and $NH_3^{(ads)}$ process dynamics become relevant for $x_3(t)$ and model parameter estimation. We have $NO_x$ and $NH_3^{(ads)}$ process dynamics as:

\begin{align*}
    x_1(k+1) &= u_1 - \pmb \phi_1^T \lr{x_3 \pmb \theta_1}\\
    x_3(k+1) &= x_3 - \pmb \phi_3^T \pmb \theta_3
\end{align*}

\itbf{Idea:} Assuming $x_3$ varies slowly, specifically at every $l$ samples, we can estimate it lumped with the constant parameters for a given window-length $l$ and then uses these parameters estimates to find the lumped parameters of the $NO_x$ and $NH_3^{(ads)}$ process dynamics.

The new-parameter vector for $NO_x$ dynamics is:
\begin{align*}
    \pmb \beta(k) &= x_3(k) \pmb \theta_1
\end{align*}

Multiplying $\theta_{1i}$ on both sides of the second equation:
\begin{align*}
    \theta_{1i} x_3(k+1) &= \theta_{1i} x_3 - \theta_{1i} \pmb \phi_3^T \pmb \theta_3\\
    \beta_i(k+1) &= \beta_i(k) - \bm{\beta_i \pmb \phi_0^T & \vline & \beta_i u_1 \pmb \phi_0 ^T & \vline & \beta_i \lr{\frac{u_2}{u_3^2 u_4^2}} \pmb \phi_0 ^T & \vline & - \theta_{1i}\lr{\frac{u_2}{u_3^2 u_4^2}} \pmb \phi_0 ^T}^T \theta_3\\
    %====
    \beta_i(k+1)&= \beta_i(k) - \bm{\beta_i \pmb \phi_0^T & \vline & \beta_i u_1 \pmb \phi_0 ^T & \vline & \beta_i \lr{\frac{u_2}{u_3^2 u_4^2}} \pmb \phi_0 ^T & \vline & - \lr{\frac{u_2}{u_3^2 u_4^2}} \pmb \phi_0 ^T}^T \theta_{\beta_i}\\
    \text{Where,} \qquad &\\
    \theta_{\beta_i} &= \mat{ \left [
                            n V \mu q_{od}  \right .&
                            n V \mu m_{od}  &
                            n V \mu c_{od}  &
                            n V \mu q_{scr} &
                            n V \mu m_{scr} &
                            n V \mu c_{scr} & \\ &
                            n V \mu b_u q_{ads} &
                            n V \mu b_u m_{ads} &
                            n V \mu b_u c_{ads} &
                            \theta_i \Gamma n V \mu b_u q_{ads} &
                            \theta_i \Gamma n V \mu b_u m_{ads} &
                            \left . \theta_i \Gamma n V \mu b_u c_{ads} \right]}
\end{align*}
